
\section*{Continuous Time Model}

The objective of this section is to untangle the dynamic impact of clicks on purchase propensities. Instead of creating shopping sessions, we create a continuous-time panel by sampling instances of purchase and nonpurchase from the event stream.

\subsubsection*{Panel Construction and Sampling}

The panel is constructed by sampling from the continuous-time universe. This process generates three distinct strata of observations, which, when combined and appropriately weighted, allow for the consistent estimation of the parameters in Equation \ref{eq:full_model}. Let the final panel be denoted by $\mathcal{P} = \mathcal{P}^+ \cup \mathcal{P}^- \cup \mathcal{P}^0$.

\paragraph{Positive Samples ($\mathcal{P}^+$).}
Let $\mathcal{C}$ be the set of all observed conversion events, where each element $c \in \mathcal{C}$ is a tuple $(i_c, v_c, t_c)$ representing a purchase by user $i_c$ from vendor $v_c$ at time $t_c$. The set of positive samples, $\mathcal{P}^+$, is constructed by creating one observation for each conversion event. For each $c \in \mathcal{C}$, we define a row in the panel where the outcome is $y_{i_ct_c} = 1$. This ensures that all conversion signals are included in the dataset. The size of this set is $|\mathcal{P}^+| = |\mathcal{C}|$.

\paragraph{Negative Samples ($\mathcal{P}^-$).}
A set of negative samples is generated to model the baseline state where conversions do not occur. Let $\mathcal{A} \subseteq \{1, \dots, N\} \times \{1, \dots, V\}$ be the set of "active" user-vendor pairs, defined as pairs with at least one ad interaction during the observation period. A total of $n^-$ negative samples are drawn. Each sample is generated by first drawing a pair $(i, v)$ uniformly with replacement from $\mathcal{A}$, and then drawing a time $t_r$ uniformly from $[0, T]$. For each such triplet $(i, v, t_r)$, a row is added to the panel with the outcome variable coded as $y_{ivt_r} = 0$. This constitutes the negative sample set $\mathcal{P}^-$, with size $|\mathcal{P}^-| = n^-$.

\paragraph{Double-Negative Samples ($\mathcal{P}^0$).}
The construction of $\mathcal{P}^+$ and $\mathcal{P}^-$ introduces sampling-induced correlation between the regressors and the outcome. The characteristics of converting users are over-represented relative to their measure in the full user-time space. To correct this, a "double-negative" sample, $\mathcal{P}^0$, is created. For every positive sample $(i_c, v_c, t_c) \in \mathcal{P}^+$, a corresponding double-negative sample is generated. This sample is an exact duplicate in its covariates---it shares the same user, vendor, timestamp, and feature vectors ($\mathbf{x}_{i_ct_c}$, $\mathbf{W}_{i_ct_c}$). However, its outcome is coded as $y_{i_ct_c} = 0$. The size of this set is $|\mathcal{P}^0| = |\mathcal{P}^+| = |\mathcal{C}|$. This ensures that the moments of the regressors, when weighted, are calculated over a sample that is representative of the entire user-time space, not one that is disproportionately populated by conversion events.

\begin{figure}[htbp!]
\centering
\includegraphics[width=\textwidth]{panel_construction_diagram.png}
\caption{Three-Strata Panel Construction. The panel combines three sampling strata: (1) Positive samples ($\mathcal{P}^+$) include all observed conversions with $y=1$ and weight $w^+=1$; (2) Negative samples ($\mathcal{P}^-$) are randomly drawn from active user-vendor pairs with $y=0$ and weight $w^- = |\mathcal{A}| \cdot T / n^-$; (3) Double-negative samples ($\mathcal{P}^0$) duplicate the covariates of positive samples but with $y=0$ and weight $w^0=-1$. The negative weight in $\mathcal{P}^0$ corrects for the over-representation of converter characteristics, ensuring weighted regressor moments are representative of the entire user-time space.}
\label{fig:panel_construction}
\end{figure}

\subsubsection*{Weights}

To recover consistent estimates of $\boldsymbol{\beta}$ from the sampled panel $\mathcal{P}$, a weighting scheme is applied. The weights are designed such that the weighted sample moments approximate the true population moments. Each observation $k \in \mathcal{P}$ is assigned a weight $w_k$.

The weight for a positive sample $k \in \mathcal{P}^+$ is unity:
\begin{equation}
w_k^+ = 1
\end{equation}
The weight for a double-negative sample $k \in \mathcal{P}^0$ is negative one:
\begin{equation}
w_k^0 = -1
\end{equation}
The weight for a negative sample $k \in \mathcal{P}^-$ is the ratio of the total measure of the sampling space to the number of negative samples drawn:
\begin{equation}
w_k^- = \frac{\mu(\mathcal{A} \times [0, T])}{n^-} = \frac{|\mathcal{A}| \cdot T}{n^-}
\end{equation}
where $\mu(\cdot)$ is the measure of the space. The estimation of Equation \ref{eq:full_model} is then performed via a weighted fixed-effects estimator that minimizes the weighted sum of squared residuals. The weighting scheme ensures that the contribution of the negative samples represents the entire non-conversion space, while the double-negative samples correct for the over-representation of converter characteristics.

The final output is a long-form panel dataset where each row corresponds to a specific point $(i, v, t)$ drawn from one of the three sampling strata. This structure is designed for the dynamic calculation of the regressors $\mathbf{x}_{ivt}$ and $\mathbf{W}_{ivt}$ at each specific timestamp and for direct use in estimation routines that can absorb high-dimensional fixed effects. Table 2 provides a schematic of the final panel structure.

\begin{table}[h!]
\centering
\caption{Structure of the Final Analysis Panel}
\begin{tabular}{p{0.2\linewidth} p{0.7\linewidth}}
\toprule
Column & Description \\
\midrule
user\_id ($i$) & Unique user identifier, used for user fixed effects $\alpha_i$. \\
vendor\_id ($v$) & Unique vendor identifier, used for vendor fixed effects $\gamma_v$. \\
week\_id & Week identifier derived from timestamp $t$, for time fixed effects $\delta_t$. \\
timestamp ($t$) & The specific, high-precision time of the sampled event. \\
outcome ($y_{ivt}$) & Binary indicator (1 for $\mathcal{P}^+$, 0 for $\mathcal{P}^- \cup \mathcal{P}^0$). \\
adstock\_... ($\mathbf{x}_{ivt}$) & A suite of dynamically calculated ad stock features. \\
control\_... ($\mathbf{W}_{ivt}$) & A suite of dynamically calculated control variables. \\
sample\_weight ($w_k$) & The statistical weight assigned to the observation. \\
\bottomrule
\end{tabular}
\label{tab:final_panel}
\end{table}

\subsubsection*{Model}

The causal effect of advertising is estimated using the three-way fixed effects specification introduced in Equation \ref{eq:full_model}. The vectors of treatment and control variables are populated with dynamically computed features derived from the user's activity history prior to time $t$. The full specification of the regression equation is:
\begin{equation}
\label{eq:full_model}
y_{ivt} = \alpha_i + \delta_t + \mathbf{W}_{v}'\boldsymbol{\phi} + \mathbf{x}_{ivt}'\boldsymbol{\beta} + \mathbf{W}_{ivt}'\boldsymbol{\theta} + \varepsilon_{ivt}
\end{equation}
The theoretical model includes a vendor fixed effect, $\gamma_v$, to control for all time-invariant vendor heterogeneity. However, due to the high cardinality of vendors relative to the number of observations, this model is not identified. We therefore substitute the vendor fixed effect with a vector of observable, time-invariant vendor characteristics, $\mathbf{W}_{v}$, such as the vendor's average advertising intensity. This serves as a proxy to control for the most salient vendor-level confounders. The primary treatment vector, $\mathbf{x}_{ivt}$, includes multiple advertising stock variables with varying exponential decay rates (e.g., 1-hour, 3-hour, 24-hour half-lives) for both impression and click events. The control vector, $\mathbf{W}_{ivt}$, includes user activity measures (e.g., auction stock, rolling counts of recent exposures), controls for time-of-day and day-of-week patterns via Fourier series, and non-linear terms (e.g., squared ad stock) to capture diminishing returns.

The model parameters are estimated using Weighted Ordinary Least Squares (WOLS), with weights applied as defined in the preceding section. The high-dimensional user ($\alpha_i$) and time ($\delta_t$, defined by week) fixed effects are not estimated directly but are absorbed using a fixed-point iteration algorithm, commonly known as the method of alternating projections. This procedure is computationally efficient and is implemented via the \texttt{pyfixest} software package. Standard errors are made robust to heteroskedasticity.

\subsubsection*{Results}

Table 3 presents the estimation results. Five specifications are shown to demonstrate the sequential impact of adding fixed effects and control variables on the parameter estimates. Model (1) is a simple weighted OLS regression. Model (2) introduces user fixed effects. Model (3) adds week fixed effects. Model (4) incorporates the core set of control variables. Model (5) represents the full specification, including multiple ad stock decay functions to flexibly model the temporal dynamics of advertising effects.

\begin{table}[h!]
\centering
\caption{Fixed Effects Regression Results for Conversion Probability}
\begin{tabular}{l S[table-format=-1.4] S[table-format=-1.4] S[table-format=-1.4] S[table-format=-1.4] S[table-format=-1.4]}
\toprule
 & {(1)} & {(2)} & {(3)} & {(4)} & {(5)} \\
 & {WOLS} & {User FE} & {User+Time FE} & {+ Controls} & {Full Model} \\
\midrule
\textit{Impression Ad Stock} & & & & & \\
\quad Ad Stock (1-day) & {0.0001$^{**}$} & {0.0001$^{*}$} & {0.0001$^{*}$} & {0.0001$^{*}$} & {0.0001} \\
& {(0.0001)} & {(0.0001)} & {(0.0001)} & {(0.0001)} & {(0.0001)} \\
\quad Ad Stock (1-hr) & & & & & 0.0010 \\
& & & & & {(0.0013)} \\
\quad Ad Stock (3-hr) & & & & & -0.0005 \\
& & & & & {(0.0009)} \\
 & & & & & \\
\textit{Click Ad Stock} & & & & & \\
\quad Ad Stock (1-hr) & {0.5710$^{***}$} & {0.5390$^{***}$} & {0.5390$^{***}$} & {0.5394$^{***}$} & {-0.5010$^{**}$} \\
& {(0.0460)} & {(0.0610)} & {(0.0610)} & {(0.0616)} & {(0.1903)} \\
\quad Ad Stock (3-hr) & & & & & {0.9450$^{***}$} \\
& & & & & {(0.1709)} \\
\quad Ad Stock (1-day) & & & & & {-0.0230$^{***}$} \\
& & & & & {(0.0042)} \\
 & & & & & \\
\textit{Controls} & & & & & \\
\quad Auction Stock (6-hr) & & & & -0.0000 & 0.0001 \\
& & & & {(0.0001)} & {(0.0002)} \\
\quad Clicks (7-day count) & & & & & {0.0004$^{***}$} \\
& & & & & {(0.0001)} \\
\midrule
Fixed Effects & & & & & \\
\quad User & {No} & {Yes} & {Yes} & {Yes} & {Yes} \\
\quad Time (Week) & {No} & {No} & {Yes} & {Yes} & {Yes} \\
\quad Vendor Controls & {No} & {No} & {No} & {Yes} & {Yes} \\
\midrule
Observations & {2,512} & {2,512} & {2,512} & {2,512} & {2,512} \\
R$^2$ Within & {--} & {0.501} & {0.501} & {0.501} & {0.613} \\
\bottomrule
\end{tabular}
\caption*{Note: The dependent variable is a binary indicator for conversion. Heteroskedasticity-robust standard errors are in parentheses. $^{***}p<0.01$, $^{**}p<0.05$, $^*p<0.1$. All models are estimated with statistical weights. Vendor controls include vendor-level average ad stock. The R² Within measures the proportion of the variation in the outcome variable that is explained by the model after accounting for the fixed effects.}
\label{tab:main_results}
\end{table}

\subsubsection*{Interpretation}

The results in Table 3 indicate a positive and statistically significant causal effect of advertising on conversion probability, although the nature of this effect is complex.

In the preferred specification with controls (Model 4), a one-unit increase in the 1-day impression ad stock is associated with a 0.013 percentage point increase in the instantaneous probability of conversion ($p < 0.1$). The corresponding effect for the 1-hour click ad stock is substantially larger, at 53.9 percentage points ($p < 0.01$). While the click coefficient is strong, it likely suffers from omitted variable bias. Specifically, while the user fixed effect $\alpha_i$ controls for a user's average intent, it cannot capture high-frequency fluctuations in intent within a given week. A click is a direct behavioral manifestation of this acute, time-varying intent, leading the model to conflate the causal effect of the ad with the pre-existing intent that drove the click. The impression effect, being less correlated with immediate user intent, may represent a more conservative and reliable causal estimate.

The full specification (Model 5) reveals a non-monotonic temporal pattern of advertising effectiveness. The inclusion of multiple, correlated ad stock regressors allows the model to approximate a flexible functional form for the decay of ad effects. For clicks, the combination of a large positive coefficient for the 3-hour decay term ($+0.945$) and negative coefficients for the 1-hour ($-0.501$) and 1-day ($-0.023$) terms suggests that the causal impact of a click does not follow a simple exponential decay. The combination of coefficients approximates a more complex, non-monotonic impulse response function, where the effect peaks in the hours following the engagement before diminishing. This finding validates the use of a continuous-time framework capable of capturing such complex temporal dynamics.

\subsubsection*{Dynamic Effects}

While the coefficients in Table 3 are informative, the temporal dynamics implied by the full model (Model 5) are best understood visually. By combining the estimated coefficients for the ad stock variables with different decay rates, we can construct the impulse response function (IRF) for a typical advertising event. The IRF traces the marginal causal effect of an ad exposure on the conversion probability over time.

Figure 1 displays the estimated instantaneous and cumulative impulse response functions for a single ad click. The left panel shows the instantaneous effect, representing the change in the conversion probability at each hour following the click. The right panel shows the cumulative effect, which is the integral of the instantaneous effect over time and represents the total number of incremental conversions generated by a single click up to that point.

\begin{figure}[h!]
    \centering
    \includegraphics[width=\textwidth]{click_impulse_response.pdf}
    \caption{Instantaneous and Cumulative Impulse Response of a Single Ad Click}
    \label{fig:irf}
    \caption*{Note: The plots are derived from the coefficients of the full specification (Model 5 in Table 3). The shaded regions represent 95\% confidence intervals calculated using the model's variance-covariance matrix. The cumulative effect is the time-integral of the instantaneous effect.}
\end{figure}

The impulse response functions provide several key insights that are not immediately apparent from the regression table alone.

The instantaneous effect of a click is non-monotonic. The model estimates a small initial effect, which then rises to a statistically significant peak approximately 1 hour after the event before decaying towards zero. This result contradicts the assumption of simple exponential decay and suggests a period of user consideration post-engagement. The initial negative coefficient for the 1-hour ad stock in the regression serves to create this delayed-peak shape.

The causal effect of a click appears to be concentrated within the first 24 hours. After this period, the instantaneous effect is not statistically distinguishable from zero, and the cumulative effect curve flattens, indicating that no further incremental conversions are being generated. This finding has direct implications for defining appropriate attribution windows.

The cumulative response function provides an estimate of the total causal impact of a single click. The function asymptotes to a value of approximately 2.68, suggesting that, on average, a single ad click generates about 2.68 incremental conversions over its lifetime. This integrated value is a more robust metric for calculating return on investment than relying on the coefficient of any single ad stock variable. It aggregates the complex temporal pattern into a single, interpretable measure of total lift.

To test for heterogeneous treatment effects, the model was augmented with an interaction term between ad stock and an indicator for top-tier vendors. The results indicate that impression advertising is substantially more effective for these vendors. The estimated coefficient for impression ad stock for top vendors is 0.037, compared to a near-zero effect of 0.00008 for other vendors ($p < 0.01$ for the difference). This suggests a strong complementarity between advertising and pre-existing brand equity; impressions are most effective when they can leverage established brand recognition.