\section{The Marketplace}

The subject of this study is a digital platform that operates as a two-sided market, connecting buyers and sellers of secondhand consumer goods \citep{rochet2003platform, rochet2006two}. Specifically, it is a consumer-to-consumer (C2C) online marketplace that primarily facilitates the exchange of apparel, supplemented by categories such as home goods and electronics. The platform's revenue model is predominantly commission-based, where it captures a share of the value of each transaction it facilitates. For sales of \$15 or more, the platform claims a 20\% commission on the final sale price; for transactions below this threshold, a flat \$2.95 fee applies. This fee structure creates a direct alignment between the platform's revenue and the gross merchandise value (GMV) exchanged between its users.

The platform facilitates transactions through two distinct mechanisms: a posted-price ``Buy Now'' option and a bilateral bargaining ``Make an Offer'' channel. The latter allows for a structured negotiation process where a potential buyer submits a binding offer. The seller can then accept, decline, or issue a counter-offer, with a successful negotiation resulting in a transaction at the agreed-upon price. This dual-path structure is common in online marketplaces for unique or differentiated goods where price discovery is a key function.

The presence of a negotiation channel incentivizes sellers to set list prices above their reservation price to create bargaining headroom. This behavior is further encouraged by platform features that facilitate targeted discounts.\footnote{One such feature, known as ``Offer to Likers,'' permits sellers to extend private, binding offers to users who have previously expressed interest in an item by 'liking' it. These offers must represent at least a 10\% discount from the public list price.} These pricing dynamics introduce a strategic dimension to the interpretation of advertising effects, as promoted listings may carry list prices intended to anchor subsequent bargaining rather than to signal a seller's true willingness-to-accept.

Separate from the formal transaction structure, the platform integrates social networking features designed to drive user engagement and organic discovery. User actions such as following other users, or liking and sharing sellers' listings, are fundamental to the user experience. A seller's organic visibility within the marketplace is, by design, partly a function of their social participation. For instance, the default search algorithm is designed to favor recently shared items, creating a direct incentive for frequent seller activity as a means of achieving organic reach. This is supplemented by features such as themed, real-time virtual shopping events and live-streaming showcases that further embed social interaction into the commerce experience.\footnote{A corporate acquisition in early 2023 precipitated a significant strategic realignment. This shift occurred as the firm consolidated its operations to North America, having ceased activities in several international markets. The new ownership, facing persistent unprofitability, prioritized the development of higher-margin monetization channels to supplement the existing transaction-based revenue stream.}

The platform offers a native advertising platform for sellers, creating a hybrid dual-revenue model. The service allows sellers to promote their listings through a sponsored search auction system, a mechanism widely used by online platforms to allocate advertising slots \citep{edelman2007internet, varian2007position}. Sellers participate by setting a weekly budget to have their listings promoted in high-visibility placements, such as search results and brand pages. The system operates on a closet-wide basis; sellers cannot manually select or exclude specific listings from promotion, with the platform's algorithm automatically determining which items to surface from a seller's entire available inventory. Promoted listings are clearly distinguished from organic results through a visual ``Promoted'' label, ensuring transparency for buyers.

The ad auction is managed by an automated, real-time bidding system. Ad rank is determined by a score that is a function of the seller's bid and a proprietary quality score, a common feature in position auctions that ensures high-ranking ads are relevant to users \citep{varian2007position}. The quality score is heavily influenced by the predicted click-through rate (pCTR) of the listing. Bids are set automatically by the platform's algorithm, which optimizes for conversions based on product characteristics (such as price) and the seller's weekly budget. This design renders sellers largely price-takers in the ad auction; apart from setting a weekly budget, they possess no granular controls such as manual bidding or daily spending caps. Budget adjustments take effect at the start of the subsequent weekly cycle, and the tool is available to all sellers without minimum participation requirements.

\begin{figure}[htbp!]
\centering
\includegraphics[width=0.85\textwidth]{figures/illustrative_diagrams/05_auction_process.pdf}
\caption{Auction process flow from user query to revenue, with autobidding formula and feedback loop.}
\label{fig:auction-process}
\end{figure}

The platform employs a 28-day, click-based attribution model to measure campaign performance. Under this model, any purchase from a seller's catalog is attributed to an advertising campaign if the buyer clicked on any of that seller's promoted listings within the preceding 28 days. This approach, sometimes referred to as a ``halo'' attribution model, credits the ad campaign with sales of items that were not directly featured in the ad that was clicked. For example, if a user clicks a promoted listing for a handbag but ultimately purchases a pair of shoes from the same seller two weeks later, the revenue from the shoe purchase is attributed to the campaign. The seller-facing dashboard reports metrics consistent with this model, including Return on Ad Spend (ROAS) and total attributed sales. To encourage adoption, the platform offers free trial periods that automatically convert to paid weekly subscriptions, presenting a natural experiment that we leverage in our analysis.

The platform's inventory is characterized by a high degree of fragmentation and a long tail, which is typical of C2C marketplaces. It consists of over 52 million unique listings, where each item is a distinct entity. The catalog is predominantly composed of apparel, accessories, and footwear, with smaller segments for home goods and media. Listings are highly descriptive, containing structured metadata such as brand, color, and size, alongside unstructured seller-provided text and images. The median listing price is \$25.00; however, the distribution is heavily right-skewed by a small number of high-value luxury goods, while the vast majority of items fall within the \$10 to \$50 price range.

Popularity is concentrated around a set of well-known mid-market and high-end brands, as well as broad style attributes. A key challenge for the platform's recommendation and advertising systems is data sparsity; given that most items are unique and have no direct purchase history, predicting conversion probabilities relies on leveraging signals from similar products and user preferences rather than item-specific historical performance.