\documentclass[11pt,a4paper]{article}
\usepackage[margin=1in]{geometry}
\usepackage{amsmath,amssymb,amsthm}
\usepackage{booktabs}
\usepackage{natbib}
\usepackage{hyperref}
\usepackage{algorithm}
\usepackage{algpseudocode}

\title{Optimization in Advertising: A Causal Inference Perspective}
\author{}
\date{}

\begin{document}

\maketitle

\section{Introduction and Notation}

In online advertising, the fundamental challenge is allocating limited budget across potential impressions to maximize business value. Traditional methods optimize for observed conversions, which include both organic conversions that would happen anyway and incremental conversions caused by advertising. This paper demonstrates via simulation why optimizing for incremental conversions yields superior performance.\footnote{All results presented are from simulation studies with synthetic data. Key assumptions: (1) negative correlation between baseline conversion and treatment effects, (2) time-varying user quality in multi-period settings, (3) diminishing returns in frequency exposure, (4) platform heterogeneity in cross-platform allocation.}

For user $j \in [M]$, let $x_j \in \{0,1\}$ denote allocation, $p_{0j}$ baseline conversion probability, $\tau_j$ treatment effect, $\text{CTR}_j$ click rate, $v_j$ value per conversion, $c_j$ cost, and $B$ total budget.\footnote{Full notation: $\tau_j = \mathbb{E}[Y_j^{(1)} - Y_j^{(0)}|X_j]$ is the causal treatment effect, $\text{CVR}_j^{\text{obs}} = p_{0j} + \tau_j$ is observed conversion rate. Throughout we use mixed-integer linear programming with binary allocation decisions.}

The critical insight is that users with high baseline conversion probability $p_{0j}$ often have low treatment effects $\tau_j$, creating negative correlation that standard methods fail to exploit.\footnote{This negative correlation arises because users likely to buy anyway require less persuasion from advertising, while those uncertain about purchasing benefit more from additional information.}

\section{Standard vs Lift-Based Optimization}

Traditional advertising systems maximize observed conversions, treating all conversions equally regardless of whether advertising caused them. This approach systematically overvalues users who would convert anyway and undervalues users whose behavior is changed by advertising.

We compare two optimization approaches, building on the online advertising allocation framework of \citet{mehta2007adwords} and the lift measurement methods of \citet{johnson2017ghost}. The standard correlation-based optimization maximizes total value from observed conversions:
$$\max_{x_j} \sum_{j=1}^M v_j \cdot \text{CTR}_j \cdot (p_{0j} + \tau_j) \cdot x_j$$
subject to the budget constraint $\sum_{j=1}^M c_j x_j \leq B$, where the choice variables are binary allocation decisions $x_j \in \{0,1\}$ for each user.

In contrast, lift-based causal optimization maximizes incremental value:
$$\max_{x_j} \sum_{j=1}^M v_j \cdot \text{CTR}_j \cdot \tau_j \cdot x_j$$
subject to the same budget constraint and choice variables. The only difference is the objective function: standard methods optimize $(p_{0j} + \tau_j)$ while causal methods optimize $\tau_j$ alone.

We generate $M = 1000$ potential impressions with realistic characteristics for our simulation study. Baseline conversions follow $p_{0j} \sim \text{Beta}(2, 8)$, yielding mean 20\% with range 5-50\%. Treatment effects are inversely related to baseline: $\tau_j = \frac{0.08}{1 + 10 \cdot p_{0j}} + \epsilon_j$ where $\epsilon_j \sim \mathcal{N}(0, 0.005)$ represents small random variation. Click rates follow $\text{CTR}_j \sim \text{Beta}(3, 20)$ with mean 13\%, costs follow CPC pricing $c_j = \text{CTR}_j \cdot (\text{Exp}(0.5) + 0.1)$, and conversion value is $v_j = 100$. This specification creates $\text{Cor}(p_0, \tau) = -0.84$, capturing the empirical pattern where users with high purchase intent need less persuasion from advertising.

With budget $B = \$20$, we solve both optimization problems using mixed-integer linear programming:

\begin{table}[h]
\centering
\begin{tabular}{lrrrr}
\toprule
Method & Users Selected & Incremental Value & Spend & iROAS \\
\midrule
Random Selection & 200 & \$82.86 & \$16.00 & 5.20× \\
Standard Optimization & 511 & \$179.11 & \$20.00 & 8.96× \\
Lift-Based Optimization & 507 & \$244.84 & \$20.00 & 12.24× \\
Perfect Information (Oracle) & 509 & \$246.57 & \$19.99 & 12.33× \\
\bottomrule
\end{tabular}
\end{table}

Lift-based optimization achieves 37\% higher iROAS than standard methods (12.24× vs 8.96×) and captures 99.3\% of oracle performance.\footnote{Oracle performance represents perfect information about treatment effects, providing an upper bound on achievable performance. All results from simulation study with synthetic data where treatment effects include realistic estimation noise.} The standard method selects similar users (511 vs 507) but generates less incremental value.

\section{Fractional Relaxation}

Binary allocation constraints make optimization computationally hard.\footnote{The binary knapsack problem is NP-complete, requiring exponential time in the worst case.} Fractional relaxation allows partial allocations $x_j \in [0,1]$, making the problem tractable while providing bounds on optimal performance.

The fractional lift-based optimization maintains the same objective of maximizing incremental value but with continuous allocations:
$$\max_{x} \sum_{j=1}^M v_j \cdot \text{CTR}_j \cdot \tau_j \cdot x_j$$
subject to $\sum_{j=1}^M c_j x_j \leq B$ where choice variables are now fractional allocations $x_j \in [0,1]$.

The optimal solution follows a threshold rule based on efficiency ratio:
$$x_j^* = \begin{cases}
1 & \text{if } \frac{v_j \cdot \text{CTR}_j \cdot \tau_j}{c_j} > \lambda^* \\
\alpha & \text{if } \frac{v_j \cdot \text{CTR}_j \cdot \tau_j}{c_j} = \lambda^* \\
0 & \text{if } \frac{v_j \cdot \text{CTR}_j \cdot \tau_j}{c_j} < \lambda^*
\end{cases}$$
where $\lambda^*$ is the optimal Lagrange multiplier and at most one user receives fractional allocation $\alpha \in (0,1)$.

Using the same 1000 users, we compare three solution methods:

\begin{table}[h]
\centering
\begin{tabular}{lrrrr}
\toprule
Solution Method & Objective Value & Fractional Users & Runtime & Gap from Binary \\
\midrule
Binary MILP & \$246.63 & 0 & 0.32s & 0.00\% \\
Fractional LP & \$246.64 & 1 & 0.42s & +0.00\% \\
Greedy Rounding & \$246.62 & 0 & 0.00s & -0.01\% \\
\bottomrule
\end{tabular}
\end{table}

The integrality gap is less than 0.01\%, indicating fractional relaxation is extremely tight. Only one user receives fractional allocation in the LP solution. Greedy rounding achieves 99.99\% of optimal with 1376× speedup, providing both an upper bound and practical algorithm.

\section{Multi-Period Budget Pacing}

Real advertising campaigns run over multiple time periods with uncertain future opportunities. Spending entire budget immediately is suboptimal; pacing ensures budget lasts while capturing high-value opportunities.

Multi-period pacing with lift optimization maximizes expected incremental conversions over $T$ periods:
$$\max_{\lambda_t} \mathbb{E}\left[\sum_{t=1}^T \sum_{j \in \mathcal{J}_t} v_j \cdot \text{CTR}_j \cdot \tau_j \cdot q_j(\lambda_t)\right]$$
subject to total expected spend constraint $\mathbb{E}\left[\sum_{t=1}^T \sum_{j \in \mathcal{J}_t} c_j \cdot q_j(\lambda_t)\right] \leq B$. The choice variables are pacing multipliers $\lambda_t \in [0,1]$ for each period, where $q_j(\lambda_t)$ is the probability of winning user $j$ and $\mathcal{J}_t$ is the set of users arriving in period $t$.

For simulation, we model a daily campaign with $T = 24$ hours where users arrive each hour with varying quality. Peak hours (8am, 12pm, 6pm) have 50\% more users with 30\% higher treatment effects: $\tau_j^{\text{peak}} = \tau_j \cdot 1.3$. With budget $B = \$20$, we compare four pacing strategies.\footnote{Uniform pacing allocates equal budget per hour, adaptive pacing weights by traffic volume, and lift-aware pacing allocates proportional to expected treatment effects in each period.}

\begin{table}[h]
\centering
\begin{tabular}{lrrrr}
\toprule
Pacing Strategy & Budget Utilization & Hours Active & iROAS & Objective Value \\
\midrule
No Pacing (Greedy) & 100.0\% & 6.7 & 5.26× & \$105.10 \\
Uniform Pacing & 99.5\% & 24.0 & 12.75× & \$253.81 \\
Adaptive Pacing & 99.6\% & 24.0 & 12.91× & \$257.11 \\
Lift-Aware Pacing & 99.6\% & 24.0 & 12.88× & \$256.48 \\
\bottomrule
\end{tabular}
\end{table}

Greedy spending depletes budget in 6.7 hours, missing high-value peak hour opportunities. Lift-aware pacing achieves 145\% higher iROAS than greedy baseline by spreading budget across the full day and prioritizing periods with higher treatment effects. All pacing strategies significantly outperform greedy allocation.

\section{Frequency Capping with Diminishing Returns}

Repeated ad exposures show diminishing returns: first impression has highest impact, subsequent ones less so. Frequency capping optimizes how many times to show ads to each user.

Frequency-capped lift optimization maximizes incremental value accounting for diminishing returns:
$$\max_{x_{jk}} \sum_{j=1}^M \sum_{k=1}^{F} v_j \cdot \text{CTR}_j \cdot \tau_j^{(k)} \cdot x_{jk}$$
subject to budget constraint $\sum_{j=1}^M \sum_{k=1}^{F} c_{jk} x_{jk} \leq B$ and frequency cap $\sum_{k=1}^{F} x_{jk} \leq F$ per user. The choice variables $x_{jk} \in \{0,1\}$ indicate whether to show the $k$-th impression to user $j$. The diminishing lift model follows $\tau_j^{(k)} = \tau_j^{(1)} / (1 + 0.6(k-1))$ representing 60\% diminishing returns, and retargeting costs increase: $c_{jk} = c_j \cdot (1 + 0.15(k-1))$.

We simulate $M = 500$ users with budget $B = \$20$, comparing naive systems that ignore diminishing returns against capping strategies:\footnote{Naive baseline uses first-impression value for all frequency decisions, a common practice when systems lack frequency response models. This leads to over-frequency on high-value users.}

\begin{table}[h]
\centering
\begin{tabular}{lrrrr}
\toprule
Frequency Strategy & Objective Value & iROAS & Avg Frequency & Users Reached \\
\midrule
No Cap (Naive) & \$184.93 & 9.35× & 10.00 & 93 \\
Hard Cap (F=3) & \$271.91 & 13.60× & 2.12 & 358 \\
Hard Cap (F=5) & \$281.47 & 14.07× & 2.49 & 339 \\
Optimal Cap & \$283.31 & 14.17× & 2.62 & 332 \\
\bottomrule
\end{tabular}
\end{table}

Naive uncapped allocation shows each user 10 times on average, reaching only 93 users and achieving 9.35× iROAS. Optimal capping (F=2-3) spreads budget across 332 users, achieving 52\% higher iROAS by avoiding wasteful over-frequency and reaching more unique users.

\section{Cross-Platform Allocation}

Advertisers distribute budgets across multiple platforms with different user bases, costs, and effectiveness. Optimal allocation must consider platform-specific lift curves.

Multi-platform lift optimization allocates budget across $K$ platforms to maximize total incremental value:
$$\max_{b_1,\ldots,b_K} \sum_{k=1}^K \max_{x_k} \sum_{j \in \mathcal{M}_k} v_{kj} \cdot \text{CTR}_{kj} \cdot \tau_{kj} \cdot x_{kj}$$
subject to total budget $\sum_{k=1}^K b_k \leq B$, per-platform budgets $\sum_{j \in \mathcal{M}_k} c_{kj} x_{kj} \leq b_k$, and binary allocations $x_{kj} \in \{0,1\}$ where $\mathcal{M}_k$ is the user set for platform $k$.

We simulate three platforms with 300 users each, differing in cost and conversion characteristics:\footnote{Search has high intent/high cost (CTR $\sim$ Beta(4,16), CPC $\sim$ Exp(0.7)+0.3), Display has low intent/low cost (CTR $\sim$ Beta(2,25), CPC $\sim$ Exp(0.3)+0.05), Video has medium characteristics (CTR $\sim$ Beta(3,22), CPC $\sim$ Exp(0.5)+0.2). All use same treatment effect structure with platform-specific baselines.}

With budget $B = \$20$, we compare allocation strategies:

\begin{table}[h]
\centering
\begin{tabular}{lrrrrr}
\toprule
Allocation Strategy & Objective Value & iROAS & Search & Display & Video \\
\midrule
Single Platform & \$92.24 & 4.61× & -- & -- & 100\% \\
Equal Split & \$179.74 & 8.99× & 33\% & 33\% & 33\% \\
Proportional & \$144.98 & 10.69× & 10\% & 38\% & 20\% \\
Optimal & \$179.15 & 8.96× & 40\% & 30\% & 30\% \\
\bottomrule
\end{tabular}
\end{table}

Single-platform allocation (Video preferred in 70\% of runs due to favorable cost structure) achieves 4.61× iROAS. Diversifying across platforms nearly doubles performance: optimal multi-platform allocation achieves 94\% higher iROAS by exploiting complementary user bases and avoiding platform-specific saturation effects.

\end{document}