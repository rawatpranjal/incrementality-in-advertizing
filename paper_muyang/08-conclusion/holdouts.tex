\section*{Analysis of Randomized User Holdouts}

This analysis estimates the impact of advertising on users by leveraging a randomized experiment implemented by the online marketplace. The marketplace assigned a portion of its user base to a holdout or control group that was withheld from receiving advertising impressions. 

\subsection*{Detection}

The first step was to identify the members of the marketplace-run control group. The defining characteristic of these users was their exclusion from ad impressions despite being active purchasers. The identification procedure began with a universe of all 4,926,305 purchasing users. A weekly attrition algorithm then iterated through 28 weeks of impression data; any user from the initial universe found to have an impression in any week was removed from the candidate set. A final cleaning step removed 3,164 holdout-assigned users who nonetheless registered click activity, resulting in a final control group of 780,969 users. The remaining 4,142,172 purchasers form the treatment group.

\begin{table}[htbp!]
\centering
\caption{Definition of Analysis Periods}
\label{tab:period_definitions}
\begin{tabular}{lll}
\toprule
Period & Time Range & Purpose \\
\midrule
Period 1 & 2025-03-10 to 2025-06-30 & Period for identifying the base population. \\
Period 2 & 2025-07-01 onward & Post-treatment period used for outcome measurement. \\
\bottomrule
\end{tabular}
\end{table}

The experiment was active for the entire sample period. To structure the analysis, the data was partitioned into Period 1 (2025-03-10 to 2025-06-30) and Period 2 (2025-07-01 onward). The analytical cohort was defined as the 1,119,128 users who made at least three purchases in Period 1. For this cohort, a rich set of control variables ($X$) were engineered by aggregating each user's activity in Period 1, including average weekly revenue, purchases, and clicks, as well as user tenure and vendor-related metrics such as vendor variety and spend concentration. Four distinct outcome variables ($Y$) were defined based on user behavior in Period 2: total revenue, total purchases, distinct vendors, and spend concentration.

A comparison of the two groups in Period 1, summarized in Table \ref{tab:covariate_balance}, reveals significant differences in user activity. The treatment group, which was exposed to advertising, demonstrates higher levels of revenue, purchases, and clicks. These differences are expected, as they reflect the effect of the advertising treatment during this period.

\begin{table}[htbp!]
\centering
\caption{Covariate Balance Check (Period 1 Averages)}
\label{tab:covariate_balance}
\begin{tabular}{lrr}
\toprule
Variable & Treatment Group & Control Group \\
\midrule
User Count & 1,111,277 & 7,851 \\
Average Revenue & \$391.34 & \$238.40 \\
Average Purchases & 9.30 & 5.91 \\
Average Clicks & 48.92 & 3.38 \\
\bottomrule
\end{tabular}
\end{table}


\subsection*{Difference-in-Means and Bias}

We estimate the effect of advertising by comparing the average outcomes for the treatment and control groups. The quantity of interest is the simple difference in means:

\begin{equation}
\Delta = E[Y_i | D_i=1] - E[Y_i | D_i=0]
\end{equation}

where $Y_i$ is the outcome for user $i$ and $D_i$ is an indicator variable equal to 1 if the user is in the treatment group (saw ads) and 0 if the user is in the control group (did not see ads). This provides a straightforward comparison of the two groups.

A significant source of bias in our estimation is the construction of the control group. The control group is composed of users who were withheld from receiving ad impressions but still made at least one purchase. This means that our sample of control group users excludes individuals who saw no ads and made no purchases. 

This exclusion likely biases our estimates of the treatment effect upwards. The non-purchasing users, if included, would lower the average purchase and revenue outcomes for the control group. As a result, the observed difference between the treatment and control groups is likely larger than the true causal effect. Our results should therefore be interpreted as an upper bound on the true effect of advertising.


\subsection*{Results}

The analysis reveals four distinct, statistically significant, and positive differences in advertising exposure on user behavior. The overall group averages for each dimension are presented in Table \ref{tab:ate_summary}.

\begin{table}[htbp!]
\centering
\caption{Group Averages and Variable Definitions}
\label{tab:ate_summary}
\begin{tabularx}{\textwidth}{l S S >{\raggedright\arraybackslash}X}
\toprule
Outcome Variable & {Absolute Lift} & {Relative Lift (\%)} & Definition \\
\midrule
Revenue (\$) & 36.34 & 43.17 & Total revenue generated by a user in Period 2. \\
\addlinespace
Purchases (\#) & 0.76 & 39.74 & Total number of purchases made by a user in Period 2. \\
\addlinespace
Vendor Variety (\#) & 0.13 & 57.74 & Number of distinct vendors from whom a user made a purchase in Period 2. \\
\addlinespace
Spend Concentration (\%) & 0.06 & 54.69 & The proportion of a user's total spend in Period 2 that is allocated to their single top vendor in that same period. \\
\bottomrule
\end{tabularx}
\end{table}

The results indicate a clear, positive association between advertising exposure and user purchasing behavior. On average, users in the treatment group, who were exposed to advertising, generated approximately 43\% more revenue and made 40\% more purchases than users in the control group. Furthermore, advertising exposure is associated with changes in shopping patterns: users exposed to ads purchased from a wider set of vendors, with vendor variety increasing by over 57\%, and also demonstrated a greater concentration of spend with their preferred vendor, with spend concentration rising by nearly 55\%.