\section{Position Effects Through Fold Discontinuities}

This section estimates the causal effect of ad rank on click-through rates using a regression discontinuity design. The setting is a mobile e-commerce marketplace where sponsored product listings appear alongside organic search results.\footnote{The platform is a peer-to-peer fashion resale marketplace with approximately 80 million users. Individual sellers pay for sponsored placement of their listings. The analysis sample covers February 2026, comprising 68,917 auctions (1 percent user sample) and 1,622 near-tie pairs at the primary bandwidth ($\tau = 0.02$) for the fold boundary.} The platform runs a real-time auction for each user query, scoring ads by predicted click-through rate times bid amount and displaying winners in descending score order.\footnote{Users search by keyword or browse product categories. Each page load triggers an auction among advertisers whose products match the query context. Winning ads appear in dedicated sponsored slots interspersed with organic results.} The key identification challenge is that higher-ranked ads differ systematically from lower-ranked ads in ways that also affect outcomes. We address this by comparing ads that just barely won versus just barely lost their position, where assignment is effectively random.

Several terms require definition. An \textit{auction} is the real-time allocation process triggered when a user loads a page; the platform evaluates eligible ads, scores them, and selects winners within milliseconds. An \textit{impression} occurs when a winning ad is rendered on the user's screen.\footnote{The platform logs an impression when the ad enters the viewport, not when it wins the auction. An ad can win a slot but never be impressed if the user does not scroll far enough. In the data, 34 percent of auction winners at rank 3 receive no impression, confirming that users frequently stop scrolling before reaching below-fold positions.} The \textit{fold} refers to the boundary of initially visible content; ads above the fold are seen without scrolling, while ads below require the user to scroll down.\footnote{The term originates from print newspapers, where content above the physical fold was visible on newsstands. In digital contexts, it refers to content visible without scrolling. The exact fold position depends on device and screen size; on the study platform, mobile users see two sponsored ads initially while desktop users see four or more.} On mobile devices, the viewport typically displays two sponsored listings at a time, making the fold boundary sharp. \textit{Autobidding} refers to the platform setting bid amounts algorithmically on behalf of advertisers, rather than advertisers choosing bids directly; this institutional feature strengthens the identification assumptions discussed below.

\begin{figure}[htbp]
\centering
\includegraphics[width=0.5\textwidth]{figures/illustrative_diagrams/02_mobile_viewport.pdf}
\caption{Mobile viewport showing the fold boundary between ranks 2 and 3. Ranks 1 and 2 appear in the initial viewport; rank 3 and below require scrolling.}
\label{fig:mobile_fold}
\end{figure}

The question of whether rank position causally affects ad outcomes faces a fundamental selection problem. Platforms rank ads by a score that combines bid and quality, so higher-ranked ads differ systematically from lower-ranked ads. As \citet{narayanankalyanam2015} observe, ``a simple mean comparison of outcomes at two positions is likely to be biased due to these selection issues.'' The platform scoring rule is $\text{AdRank}_i = \text{bid}_i \times \text{QualityScore}_i$, so ads in position 1 have higher AdRank than ads in position 2, but ads with higher AdRank may also be intrinsically more clickable. This confounding of position effects with selection effects invalidates naive comparisons of outcomes across ranks.

% Three standard approaches to this problem have known limitations. Experimentation that randomizes the focal advertiser's bids does not eliminate bias induced by competitors' strategic bidding behavior.\footnote{Consider a competing advertiser who bids higher on days when it expects elevated sales due to a promotion. The promotion may simultaneously lower sales at the focal advertiser and push the focal advertiser to a lower position through higher competitor bids. Even absent a true position effect, this negative correlation between position and sales would appear spuriously as a position effect. Randomizing the focal advertiser's bids cannot eliminate bias induced by competitors' unobserved strategic behavior. Large-scale randomization of all advertisers' positions is typically infeasible because search engines require advertiser consent and forgo revenue on experimental pages.} Instrumental variables are difficult because demand-side factors correlated with position typically cannot be excluded from consumer outcomes.\footnote{Valid instruments must correlate with position but not directly affect consumer behavior. Demand-side factors such as consumer intent or purchase propensity are correlated with position through the auction mechanism but cannot be excluded from outcomes. Cost-side instruments that vary with position are difficult to identify. Latent instrumental variable approaches assume normality of outcomes, but click and sales distributions are highly non-normal due to periodic promotions and sparse events. Such approaches also assume selection effects do not vary across positions, contradicting the local nature of auction-induced selection.} Parametric selection models require the correct functional form for complex auction mechanisms, yet selection effects are likely to be highly local and may vary by position with unpredictable signs.\footnote{Parametric approaches jointly estimate outcome and position equations, modeling selection through error correlations. This requires a correctly specified position equation, but position is determined through complex auction processes involving own bidding rules, competitor behavior, and platform algorithms. Selection mechanisms differ across positions: at position 1, the focal advertiser competes against a different set of competitors than at position 5. Furthermore, multiple mechanisms may operate simultaneously with opposing signs, making global parametric assumptions untenable. The approach also requires exclusion restrictions that are difficult to justify in this context.} \citep{narayanankalyanam2015}

The regression discontinuity design addresses this selection problem by exploiting the discrete nature of position assignment among advertisers whose scores are nearly identical. The key insight is comparing outcomes when an advertiser ``just barely won'' the bid to situations when the advertiser ``just barely lost'' the bid. Occasions on either side of the threshold can be considered equivalent in terms of underlying propensities, and any difference can be attributed to position alone.\footnote{This identification strategy has been applied to estimate consumer surplus from Uber's discrete surge pricing, where users with nearly identical underlying demand conditions face discretely different prices \citep{cohen2016uber}, and to estimate the causal effect of Yelp star ratings on restaurant revenue, where establishments with nearly identical true ratings receive discretely different displayed ratings due to rounding thresholds \citep{luca2016,andersonmagruder2012}.}

Let $a$ index auctions and $i$ index ads. The platform score is $s_{ai} = q_{ai} \times b_{ai}$, where $q_{ai}$ is the predicted click-through rate and $b_{ai}$ is the final bid. Ads are ranked in descending order of score. For bidders with ranks $r$ and $r+1$, the forcing variable is the score difference $\Delta s = s_r - s_{r+1}$, and the treatment threshold is zero. Define the normalized score gap $z = (s_r - s_{r+1})/s_r$ and include pairs where $z \leq \tau$ for a small bandwidth $\tau$; such pairs are called \textit{near-ties} because their scores differ by less than $\tau$ percent. Within each pair, denote by $\text{Lucky}$ the higher-scoring ad and by $\text{Unlucky}$ the lower-scoring ad. When $z \approx 0$, assignment is as-if random: both ads faced identical user, query, and competitive context; their scores are nearly equal, matching ad appeal; and the only source of variation is the discrete position assignment.\footnote{Among users who saw the higher-ranked ad but did not scroll, only 2.5 percent clicked on it. The remaining 97.5 percent disengaged from the sponsored results entirely, suggesting scroll propensity reflects user-level engagement rather than ad-specific response.}

A distinctive feature of the advertising setting is that observation of outcomes depends on the user's scroll behavior. Among near-tie pairs where the higher-ranked ad was impressed, the probability of scrolling to see the lower-ranked ad ranges from 54.6 percent at the fold boundary to 89.0 percent at deeper boundaries. This sequential exposure could threaten identification if scroll propensity correlated with the score gap. We test this directly: regressing scroll indicators on the normalized score gap $z$ yields coefficients indistinguishable from zero at all boundaries ($p > 0.1$). Consistent with the \citet{weitzman1979} sequential search framework, the stopping decision is independent of ad quality conditional on the near-tie restriction.

The local linear regression specification follows \citet{narayanankalyanam2015}:
\begin{equation}
Y_{ai} = \alpha + \beta \cdot \text{Lucky}_{ai} + \gamma_1 z_{ai} + \gamma_2 z_{ai} \cdot \text{Lucky}_{ai} + \varepsilon_{ai}
\label{eq:rdd_spec}
\end{equation}
where $Y$ is the outcome, $\text{Lucky}$ indicates the higher-scoring ad, and $z$ is the normalized score gap. The coefficient $\beta$ is the local average treatment effect of position at the threshold. The terms $\gamma_1 z$ and $\gamma_2 z \cdot \text{Lucky}$ control for local linear trends in the forcing variable on each side of the cutoff, following the recommendation to allow different slopes above and below the threshold. Standard errors are clustered at the pair level to account for within-auction correlation.\footnote{Following the RD framework, the local average treatment effect is the limiting difference in expected outcomes on either side of the threshold: $\beta = \lim_{\lambda \to 0} \mathbb{E}[Y \mid z = \lambda] - \lim_{\lambda \to 0} \mathbb{E}[Y \mid z = -\lambda]$.}

The validity of the RD design requires that the forcing variable is continuous at the threshold and that agents cannot precisely manipulate their score to land just above the cutoff. In our setting, the validity conditions are strengthened by the platform's autobidding mechanism. The bid formula is $\text{FINAL\_BID} = (\text{pCVR} \times \text{AOV})/\text{target\_ROAS}$, where pCVR is predicted by the platform's machine learning models and AOV is approximately equal to product price. Advertisers do not control bids directly; the platform determines them algorithmically. This eliminates strategic manipulation concerns that arise in second-price auctions where advertisers set their own bids. The score variation among near ties is driven by small differences in ML predictions, which are as-if random conditional on the features used by the model.\footnote{This is a stronger validity condition than that in \citet{narayanankalyanam2015}, who rely on the second-price auction structure to argue that advertisers lack incentive to target being just above a threshold. With autobidding, there is literally no advertiser strategic behavior to consider.}

The analysis proceeds in two stages that decompose the total position effect using equation~\eqref{eq:rdd_spec}. The first stage examines the extensive margin by estimating the effect on impression probability, where $Y = 1$ if the ad was shown to the user. The coefficient $\hat{\beta}_{\text{exp}}$ isolates the visibility effect in a narrow score neighborhood. The second stage conditions on both items being exposed and estimates the effect on click probability among pairs where both ads were impressed. The coefficient $\hat{\beta}_{\text{ctr}}$ separates the visibility channel from any within-exposure persuasion effect. The decomposition distinguishes whether position operates through being shown to the user or through increased clicking conditional on being seen.\footnote{Impressions and clicks are mapped to candidate ads by the composite key of auction, product, and vendor. This ensures that exposure and clicks are attributed to the paired ads within the same auction context.}

\begin{table}[htbp]
\centering
\caption{Exposure and click effects by rank boundary at $\tau = 0.02$}
\label{tab:rdd_main}
\small
\begin{tabular}{lcccccccr}
\toprule
Boundary & $\hat{\beta}_{\text{exp}}$ & (SE) & $\hat{\beta}_{\text{click}}$ & (SE) & $\hat{\beta}_{\text{ctr}}$ & (SE) & $N$ \\
\midrule
2 vs 3 (fold) & 0.235*** & (0.027) & 0.005 & (0.006) & 0.022 & (0.016) & 1622 \\
4 vs 5 & 0.044* & (0.019) & 0.007 & (0.005) & 0.023 & (0.014) & 2519 \\
6 vs 7 & 0.027* & (0.013) & 0.004 & (0.004) & 0.018 & (0.014) & 3228 \\
7 vs 8 (placebo) & $-$0.010 & (0.008) & 0.000 & (0.004) & $-$0.003 & (0.013) & 3496 \\
\bottomrule
\multicolumn{8}{p{0.95\textwidth}}{\footnotesize Notes: $\hat{\beta}_{\text{exp}}$ = effect on impression probability; $\hat{\beta}_{\text{click}}$ = effect on click probability (unconditional); $\hat{\beta}_{\text{ctr}}$ = effect on click probability conditional on both ads impressed. Standard errors clustered by pair. *$p<0.05$, **$p<0.01$, ***$p<0.001$.}
\end{tabular}
\end{table}

\begin{table}[htbp]
\centering
\caption{Bandwidth sensitivity at the fold boundary (2 vs 3)}
\label{tab:rdd_bandwidth}
\begin{tabular}{lcccccccc}
\toprule
$\tau$ & $\hat{\beta}_{\text{exp}}$ & (SE) & $\hat{\beta}_{\text{click}}$ & (SE) & $\hat{\beta}_{\text{ctr}}$ & (SE) & $N$ \\
\midrule
0.005 & 0.189*** & (0.047) & 0.010 & (0.008) & 0.070* & (0.035) & 661 \\
0.010 & 0.197*** & (0.037) & 0.005 & (0.007) & 0.040 & (0.023) & 999 \\
0.020 & 0.235*** & (0.027) & 0.005 & (0.006) & 0.022 & (0.016) & 1622 \\
0.050 & 0.260*** & (0.018) & 0.007 & (0.004) & 0.014 & (0.010) & 3351 \\
\bottomrule
\multicolumn{8}{l}{\footnotesize Notes: *$p<0.05$, **$p<0.01$, ***$p<0.001$. Local linear regression with pair-clustered SEs.}
\end{tabular}
\end{table}

\begin{table}[htbp]
\centering
\caption{Device heterogeneity at the fold boundary ($\tau = 0.02$)}
\label{tab:rdd_device}
\begin{tabular}{lcccccccc}
\toprule
Device & $\hat{\beta}_{\text{exp}}$ & (SE) & $\hat{\beta}_{\text{click}}$ & (SE) & $\hat{\beta}_{\text{ctr}}$ & (SE) & $N$ \\
\midrule
Pooled & 0.235*** & (0.027) & 0.005 & (0.006) & 0.022 & (0.016) & 1622 \\
Mobile & 0.436*** & (0.036) & --- & --- & 0.037 & (0.024) & 1069 \\
Desktop & 0.085* & (0.033) & --- & --- & $-$0.001 & (0.012) & 232 \\
\bottomrule
\multicolumn{8}{l}{\footnotesize Notes: Device inferred from impression batch size (mobile $\leq 2$, desktop $\geq 3$). Mobile}\\
\multicolumn{8}{l}{\footnotesize comprises 81\% of auctions. *$p<0.05$, **$p<0.01$, ***$p<0.001$.}
\end{tabular}
\end{table}

\begin{table}[htbp]
\centering
\caption{Placement heterogeneity at the fold boundary ($\tau = 0.02$)}
\label{tab:rdd_placement}
\begin{tabular}{lcccccccc}
\toprule
Placement & $\hat{\beta}_{\text{exp}}$ & (SE) & $\hat{\beta}_{\text{click}}$ & (SE) & $\hat{\beta}_{\text{ctr}}$ & (SE) & $N$ \\
\midrule
Search (P1) & 0.235*** & (0.027) & 0.005 & (0.006) & 0.022 & (0.016) & 1622 \\
Brand (P2) & 0.301*** & (0.038) & $-$0.005 & (0.012) & $-$0.033 & (0.027) & 847 \\
Product (P3) & 0.047*** & (0.009) & 0.001 & (0.002) & 0.027 & (0.027) & 3438 \\
Category (P5) & 0.021* & (0.009) & $-$0.003 & (0.002) & $-$0.047 & (0.032) & 1394 \\
\bottomrule
\multicolumn{8}{l}{\footnotesize Notes: *$p<0.05$, **$p<0.01$, ***$p<0.001$. Local linear regression with pair-clustered SEs.}
\end{tabular}
\end{table}

Table~\ref{tab:rdd_main} presents the main results. At the fold boundary of rank 2 versus 3, the exposure effect is $\hat{\beta}_{\text{exp}} = 0.235$ (SE = 0.027, $p < 0.001$), yet the unconditional click effect is essentially zero ($\hat{\beta}_{\text{click}} = 0.005$, SE = 0.006). The conditional click effect among pairs where both ads were impressed is also indistinguishable from zero. This pattern---large exposure effects but null click effects---persists at deeper boundaries: exposure effects of 0.044 at ranks 4 versus 5 and 0.027 at ranks 6 versus 7, with click effects near zero throughout. The placebo boundary of 7 versus 8, where both positions are below the fold, shows zero effect on all margins, as expected if the mechanism is visibility rather than some spurious correlation with score. Table~\ref{tab:rdd_bandwidth} reports sensitivity to the bandwidth choice at the fold. The exposure effect is highly significant across all bandwidths, ranging from 0.189 at $\tau = 0.005$ to 0.260 at $\tau = 0.05$ ($p < 0.001$ throughout), while the unconditional click effect remains insignificant at all bandwidths. Table~\ref{tab:rdd_device} reveals substantial heterogeneity across device types. The fold effect is substantially larger on mobile devices ($\hat{\beta}_{\text{exp}} = 0.436$, SE = 0.036, $p < 0.001$) than on desktop ($\hat{\beta}_{\text{exp}} = 0.085$, SE = 0.033, $p < 0.05$). This pattern is consistent with the viewport architecture: mobile users see only two ads above the fold whereas desktop users see four or more, making the 2 versus 3 boundary a true fold on mobile but not on desktop. Mobile comprises 81 percent of auctions in the sample. In all cases, the conditional click effect remains indistinguishable from zero.\footnote{Device type is inferred from impression batch sizes, where a batch is impressions with identical auction and timestamp. Mobile viewports show at most two ads simultaneously, while desktop shows three or more. The local linear specification in equation~\eqref{eq:rdd_spec} includes interaction terms to allow different slopes on each side of the threshold.}

Table~\ref{tab:rdd_placement} reports heterogeneity across placement types. Brand pages exhibit the strongest fold effect ($\hat{\beta}_{\text{exp}} = 0.301$, SE = 0.038), consistent with users arriving at a brand page with focused intent and minimal scrolling. Search pages show a similar effect of 0.235, reflecting the mixed intent of users browsing search results. Product pages have a smaller but significant effect ($\hat{\beta}_{\text{exp}} = 0.047$, SE = 0.009), consistent with the below-fold placement of sponsored listings on product detail pages, where users scroll past primary content before encountering ads. Category pages exhibit the smallest effect ($\hat{\beta}_{\text{exp}} = 0.021$, SE = 0.009). The conditional click effect is indistinguishable from zero across all placements, reinforcing the interpretation that position operates through visibility rather than persuasion regardless of page context.

The pattern indicates that position operates through visibility rather than persuasion, but the additional impressions do not translate into additional clicks. At the fold, moving from rank 3 to rank 2 increases the probability of being seen by 24 percentage points but does not increase the probability of being clicked---either unconditionally or conditional on being seen. The unconditional click effect of 0.5 percentage points is economically negligible and statistically indistinguishable from zero. This implies that the marginal impressions gained from above-fold placement are low quality: they reach users who view but do not engage. Users who would click are likely those who scroll regardless of fold position. The result that both conditional and unconditional click effects are zero suggests that the position effect is purely a visibility threshold with no downstream conversion benefit. The contribution is not demonstrating that above-fold ads receive more impressions---this is mechanical---but rather showing that this visibility advantage does not translate into clicks under conditions that permit causal inference.

Four diagnostics support the identifying assumptions. Balance tests confirm that quality scores and bids are similar within pairs (no selection on observables). Density tests show smooth mass in the score gap without bunching at zero (no manipulation). Rank-score alignment verifies that 85 to 90 percent of pairs have higher-scoring ads in better positions (the platform ranks as claimed). The placebo boundary at ranks 7 versus 8 shows zero effect (both positions are below fold, so no visibility difference expected).
