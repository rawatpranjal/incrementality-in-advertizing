\subsection*{Confounding Variables}

When studying advertising effectiveness in this marketplace, several confounders simultaneously influence both ad exposure and purchase outcomes. Most importantly, we lack data on photo quality—a decisive factor in a visual marketplace centered on fashion and apparel. High-quality, well-lit images drive both click-through rates and conversion, yet our dataset contains only text and structured metadata. Similarly, while we observe product prices, their interpretation is complicated by sellers' strategic behavior: many list items above reservation prices to create negotiation room, meaning the posted price signals both market positioning and bargaining strategy. The product's category and brand strongly predict both algorithmic promotion and organic demand. Listings from popular brands attract more attention regardless of advertising. Other product attributes—whether an item is new with tags or used, photo count, and posting recency—also shape both the ad algorithm's prioritization and buyers' purchase propensities.

User-level factors also confound the relationship. The platform's retargeting algorithms target frequent purchasers more aggressively, but these users also have intrinsically higher purchase rates. Users who spend more time browsing see more ads through exposure, and this session depth correlates with purchase likelihood. Search query specificity is particularly consequential: someone searching for a specific brand and size exhibits high purchase intent and triggers narrow, targeted ad placements. Yet we observe only the auction generated, not the query text itself. Users who save items to their favorites list receive retargeting for those products while already being further along in their consideration journey.

On the seller side, vendors with larger inventories naturally have more items available for promotion and more products attributable under the platform's closet-wide model. This marketplace has distinctive confounding through social engagement. Vendors who actively share listings, participate in themed shopping events, and build follower bases gain organic visibility—visibility that operates independently of, and potentially competes with, paid advertising. A vendor's pricing strategy, particularly willingness to negotiate or offer discounts, affects both ad click-through rates and conversion. These seller-level factors interact with temporal patterns: shopping activity fluctuates predictably across days and times, affecting both when the platform delivers ads and when users purchase. Seasonal demand creates correlated surges in advertising and buying—winter apparel campaigns intensify as consumers seek cold-weather clothing. User-vendor relationships add a final dimension: buyers who previously purchased from a seller are both more likely to be retargeted with that seller's ads and more inclined toward repeat purchases due to established trust.
