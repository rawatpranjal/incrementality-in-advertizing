\subsection*{Optimization and Mechanisms}

While the empirical literature establishes the importance of measuring advertising effectiveness, a parallel stream of work develops the optimization frameworks and mechanisms that underpin modern advertising platforms. This section reviews the core optimization problems that arise in automated bidding systems, budget allocation, and auction design.

The foundational problem in online advertising is budgeted allocation, where impressions indexed by $j$ arrive sequentially and must be assigned to advertisers $i$ subject to budget constraints $B_i$. The objective maximizes total value $\sum_{ij} v_{ij} x_{ij}$ where $x_{ij} \in \{0,1\}$ indicates whether impression $j$ is allocated to advertiser $i$, subject to the constraint that each advertiser's total spend $\sum_j c_{ij} x_{ij}$ does not exceed budget $B_i$. \citet{mehta2007adwords} formalize this as the AdWords problem and develop a primal-dual algorithm achieving a competitive ratio of $1-1/e$ under random arrival order, establishing the theoretical limits of online allocation.\footnote{Under adversarial arrivals, no deterministic algorithm exceeds $1/2$-competitive, demonstrating the importance of stochastic assumptions in real-world settings.} \citet{mehta2013online} provide a comprehensive survey unifying the allocation, pacing, and auction design problems that emerge in this context. Extensions by \citet{devanur2016online} generalize the framework to handle complex budget structures arising in practice, including hierarchical budgets and multiple resource constraints. These allocation mechanisms form the backbone of modern ad exchanges, determining which ads are shown to which users under finite advertiser budgets.

When advertisers participate in repeated auctions over time indexed by $t \in [T]$, the optimization problem shifts from one-shot allocation to dynamic pacing. Letting $q_t(b)$ denote the probability of winning auction $t$ with bid $b$, $p_t(b)$ the expected payment, and $v$ the advertiser's value per conversion, the objective maximizes expected utility $\mathbb{E}[\sum_t q_t(\lambda v)(v - p_t(\lambda v))]$ where $\lambda \in [0,1]$ is the pacing multiplier, subject to the budget constraint $\mathbb{E}[\sum_t p_t(\lambda v)] \leq B$. \citet{balseiro2017learning} analyze learning in repeated auctions with budgets, deriving optimal pacing multipliers that balance the tradeoff between exhausting budgets too quickly and missing valuable impressions. Their regret bounds establish how quickly an autobidding agent can learn to pace effectively without prior knowledge of the auction environment. \citet{conitzer2022pacing} characterize pacing equilibria in first-price auction markets, showing how budget-constrained bidders shade their bids below their values in equilibrium, with the degree of shading determined by the shadow price of the budget constraint. The practical implementation of these ideas is demonstrated by \citet{karande2022optimal}, who formulate spend-rate estimation as a joint forecasting and control problem, and \citet{chen2024optimization}, who describe the deployment of optimization-based pacing at eBay, comparing throttling and bid-shading approaches to budget management.\footnote{\citet{feldman2010budget} provide an earlier treatment focusing on search advertising, establishing convexity conditions under which gradient-based pacing algorithms converge to optimal spend rates.}

An alternative formulation models autobidding as a constrained Markov decision process, where states $s_t$ capture auction characteristics, actions $a_t$ represent bids, and the policy $\pi$ maps states to actions. The objective maximizes expected reward $\mathbb{E}_\pi[\sum_t r(s_t, a_t)]$ where $r(\cdot)$ is the reward function, subject to the cost constraint $\mathbb{E}_\pi[\sum_t c(s_t, a_t)] \leq B$ where $c(\cdot)$ is the cost function and $B$ is the budget. \citet{cai2017real} pioneer the application of reinforcement learning to real-time bidding in display advertising, framing the problem as a sequential decision task where the agent learns a bidding policy from historical auction outcomes. \citet{wu2018budget} extend this to budget-constrained settings using model-free RL, while \citet{zhou2025ocpc} describe a production system at scale implementing constrained MDP methods for cost-per-click optimization. These approaches are closely related to the bandits with knapsacks framework introduced by \citet{badanidiyuru2013bandits}, where an agent repeatedly selects from a finite set of arms $k \in [K]$, earning stochastic rewards $r_k$ while consuming resources $c_k^{(d)}$ from multiple budgets $B^{(d)}$ indexed by dimension $d \in [D]$. \citet{badanidiyuru2018bandits} establish regret bounds of $O(\sqrt{KDT \log T})$ where $K$ is the number of arms, $D$ is the number of resource dimensions, and $T$ is the time horizon. \citet{agrawal2016linear} generalize the framework to contextual settings, achieving efficient learning with confidence ellipsoid methods when rewards and costs are linear in features. The key insight across these methods is the use of Lagrangian relaxation to convert constrained optimization into unconstrained reward maximization with shadow prices on the constraints.

Advertisers often impose performance constraints beyond simple budget limits, such as maximum cost per acquisition or minimum return on investment. Letting $y_j$ denote conversions from impression $j$, $p_j$ the payment, and $\tau$ the target CPA, the objective maximizes total conversions $\sum_j q_j y_j$ where $q_j \in [0,1]$ is the allocation probability, subject to the CPA constraint $\sum_j p_j / \sum_j q_j y_j \leq \tau$. For ROI constraints with value per conversion $v_j$ and target return $\rho$, the constraint becomes $\sum_j v_j q_j y_j / \sum_j p_j \geq \rho$. \citet{despotovic2018cost} formalize cost-per-action constrained auctions, showing how platforms can elicit truthful CPA targets from advertisers and run efficient mechanisms that allocate impressions to maximize conversions subject to these constraints. \citet{chen2019cpa} develop a bidding framework that decomposes the CPA constraint into per-impression decisions using Lagrangian multipliers. \citet{deng2023multichannel} address the multi-platform setting where an advertiser must allocate budget $b_k$ across multiple advertising channels $k \in [K]$ subject to a global ROI constraint, proving convergence guarantees for their algorithm under standard concavity assumptions on the response functions. These methods extend classical budget pacing to settings where the constraint is on average cost per outcome rather than total spend.

A distinct line of research focuses on bidding to maximize incremental effects rather than observed conversions. Denoting the treatment indicator by $T_j \in \{0,1\}$ and potential outcomes by $Y_j^{(1)}$ and $Y_j^{(0)}$, the objective maximizes total incremental lift $\sum_j \tau_j T_j$ where $\tau_j = \mathbb{E}[Y_j^{(1)} - Y_j^{(0)} | X_j]$ is the conditional average treatment effect given covariates $X_j$, subject to the budget constraint $\sum_j p_j T_j \leq B$. \citet{moriwaki2020unbiased} introduce lift-based bidding, where the autobidding system estimates the causal effect of showing an ad and bids in proportion to this incremental lift rather than the raw conversion probability. This addresses the fundamental selection problem that users with high baseline conversion rates are not necessarily those for whom advertising is most effective. They develop debiasing methods based on inverse propensity scoring to correct for the fact that treatment assignment is non-random in observational auction data. \citet{moriwaki2022real} describe the real-world deployment of this system, detailing the engineering challenges of estimating heterogeneous treatment effects at scale and integrating these estimates into the bidding logic. \citet{dikkala2019online} formalize online causal optimization in real-time bidding auctions, deriving learning algorithms that combine policy optimization with doubly robust treatment effect estimation from logged data. This connects the optimization literature to the empirical measurement literature discussed earlier, as the effectiveness of lift-based bidding depends critically on the quality of the causal estimates.

Guaranteed delivery contracts introduce a different class of optimization problems, where the platform must allocate impressions indexed by $j$ to contracts $i$ with allocation variables $x_{ij} \in \{0,1\}$. The objective maximizes total value $\sum_{ij} v_i x_{ij}$ subject to capacity constraints $\sum_i x_{ij} \leq 1$ ensuring each impression goes to at most one contract, and delivery constraints $\sum_j x_{ij} \geq G_i$ ensuring each contract $i$ receives at least $G_i$ guaranteed impressions. \citet{vee2010ad} formulate ad serving with guaranteed delivery as a flow problem on a bipartite graph, developing a compact allocation plan that satisfies delivery guarantees while respecting user-level frequency caps.\footnote{Frequency capping is itself an optimization problem; \citet{buchbinder2011frequency} show that online assignment with frequency constraints admits competitive algorithms using multiplicative weights methods, while \citet{gao2025reach} extend reach optimization to settings with privacy constraints.} \citet{hojjat2017unified} provide a unified framework encompassing both reach and frequency requirements, formulating the scheduling problem as a mixed-integer program and deriving dual-based allocation heuristics. \citet{chakrabarti2012traffic} address the traffic shaping problem: when guaranteed contracts risk underdelivery, the platform can influence which users visit the site through content recommendation, jointly optimizing both ad allocation and user traffic patterns to minimize delivery shortfalls. These methods are essential for direct-sold advertising where contracts specify delivery targets rather than budget limits.

Auction design questions concern the platform's choice of pricing rule and reserve prices. For a reserve price $r$ with $N$ bidders having values drawn from distribution $F$, the objective maximizes expected revenue $R(r) = r(1-F(r))^N + N \int_r^\infty (1-F(v))^{N-1} v f(v) dv$, balancing revenue when the highest bid equals the reserve against revenue when bids exceed it. For bidders in first-price auctions with value $v$ and bid $b$, the objective maximizes surplus $(v-b) \cdot \Pr(\text{win}|b)$ where the probability of winning depends on the distribution of competing bids. \citet{myerson1981optimal} establishes the foundational result that revenue-maximizing auctions set reserve prices according to the virtual value condition, which depends on the distribution of bidder values. \citet{feng2021reserve} develop methods for learning optimal reserve prices in first-price auctions from historical data, addressing the complication that observing only winning bids creates a censored sample. \citet{choi2020optimal} study reserve pricing in name-your-own-price auctions, estimating consumer value distributions and deriving optimal reserves. For first-price auctions, where bidders must shade their bids below their values, \citet{chen2021deep} develop deep learning methods to estimate the distribution of competing bids and compute optimal shading factors, while \citet{wang2024double} introduce distributionally robust approaches that account for uncertainty in both the value and bid distributions. \citet{balseiro2023learning} analyze learning to bid in first-price auctions when bidders face budget constraints, showing how the presence of budgets alters the equilibrium bid shading and deriving algorithms with provable convergence guarantees.

Finally, several works address cross-platform budget allocation and robust optimization under uncertainty. For multi-platform allocation with platforms indexed by $k \in [K]$ and budget allocation $b_k$ to each platform, the objective maximizes total conversions $\sum_k C_k(b_k)$ where $C_k(\cdot)$ is the conversion response function for platform $k$, subject to the budget constraint $\sum_k b_k \leq B$. For robust pacing under distributional uncertainty, the objective maximizes worst-case expected value $\min_{\mathcal{D} \in \mathcal{U}} \mathbb{E}_\mathcal{D}[\sum_t q_t v_t]$ over an uncertainty set $\mathcal{U}$ of possible distributions, subject to probabilistic budget constraints $\Pr_\mathcal{D}(\sum_t p_t \leq B) \geq 1-\delta$ that must hold for all distributions in the uncertainty set. \citet{hasu2021stochastic} formulate multi-platform budget optimization as a stochastic bandit problem, where an advertiser must allocate budget across multiple advertising channels with unknown and time-varying response functions, achieving regret bounds that scale with the number of platforms. \citet{grigas2017profit} study the demand-side platform's problem of allocating multiple advertisers' budgets across real-time bidding opportunities to maximize platform profit, deriving Lagrangian decomposition methods that scale to production settings. Robustness concerns arise because advertising response is highly variable and forecasts are noisy; \citet{balseiro2023robust} develop distributionally robust pacing algorithms that provide performance guarantees even when the distribution of future auction opportunities differs from historical data, requiring only a single sample from the distribution to construct conservative pacing policies. \citet{zhao2022mcmf} address delayed feedback in conversion prediction, developing multi-constraint methods that stabilize bid control when conversion labels arrive days after clicks. \citet{liu2024delayed} extend this to settings with very long delays, using survival models to predict ultimate conversions from partially observed outcomes. These robustness and delay-aware methods acknowledge that the optimization must be performed under significant uncertainty about future outcomes and with incomplete information about past actions.
