\section{The Endogeneity of Ad Position}

A central question in the economics of online advertising is to quantify the causal effect of an advertisement's position on its performance. A naive ordinary least squares regression of a performance metric, such as click-through rate, on ad rank typically reveals a strong negative correlation: ads at the top of the page receive significantly more clicks than ads at the bottom. However, this correlation is unlikely to represent a causal relationship due to the endogenous nature of ad rank.

Ad auction systems are designed to award the most prominent positions to the ads that are most likely to be successful. Ad rank is a function of both the advertiser's bid and a quality score, which itself is often a prediction of the ad's relevance and click-through rate. Consequently, higher-quality, more appealing advertisements are systematically selected into higher ranks. This creates a selection problem: a simple regression cannot disentangle the causal effect of being in a better position from the fact that better ads are placed in those positions.

To address this challenge, we employ complementary empirical strategies that isolate the causal effect of position from confounding by ad appeal and context. We first estimate within-surface position gradients controlling for quality and fixed effects, and then implement a near-tie design that forms within-auction local comparisons around fold boundaries. The latter leverages the deterministic score-based ranking rule to construct as-if randomized contrasts at the margin where visibility changes discretely. Throughout, we report diagnostics for covariate balance, continuity, and rank–score alignment, and we separate the extensive margin of exposure from within-exposure click differences.

