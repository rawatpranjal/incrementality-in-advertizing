
We construct a vendor-week panel to analyze the effectiveness of advertising from the perspective of the firms selling on the marketplace. This panel aggregates data at the vendor level, allowing us to model the relationship between a vendor's advertising activities and their revenue.

The vendor-week panel comprises nearly one million observations from over 150,000 unique vendors across 26 weeks. Table \ref{tab:vendor_panel_dims} provides a high-level overview of the panel's dimensions.

\begin{table}[htbp!]
\centering
\caption{Vendor-Week Panel Dimensions}
\label{tab:vendor_panel_dims}
\begin{tabular}{lr}
\toprule
Metric & Value \\
\midrule
Total Observations (Rows) & 979,290 \\
Unique Vendors & 150,075 \\
Unique Weeks & 26 \\
\bottomrule
\end{tabular}
\end{table}

The panel is notably unbalanced, a common feature of firm-level data where a large portion of vendors are observed for only a few weeks. This attrition poses estimation challenges, though our fixed-effects and empirical Bayes methods are robust to this structure. For instance, over 50 percent of vendors appear in the panel for five or fewer weeks, while only a small fraction are present for the entire 26-week sample period. %\ TODO: Fill in precise percentages for vendor observation distribution.

The distributions of clicks and revenue are highly skewed, with 42.6\% of vendor-weeks having zero revenue. Table \ref{tab:vendor_dist} presents the overall distribution of the key metrics.

\begin{table}[htbp!]
\centering
\caption{Distribution of Key Metrics per Vendor-Week}
\label{tab:vendor_dist}
\begin{tabular}{lrrr}
\toprule
Statistic & Clicks & Purchases & Revenue (\$) \\
\midrule
Mean & 148.78 & 2.11 & 88.02 \\
Std. Dev. & 211.42 & 6.14 & 380.85 \\
Min & 1 & 0 & 0.00 \\
25\% & 56 & 0 & 0.00 \\
50\% & 102 & 1 & 18.00 \\
75\% & 168 & 2 & 76.00 \\
Max & 11,292 & 1,288 & 772.81 \\
\bottomrule
\end{tabular}
\end{table}

Platform-wide aggregates show a clear expansion over the 26-week sample period, which motivates the inclusion of week fixed effects ($\gamma_t$) to absorb common time-based shocks. The total number of active vendors, gross merchandise value, and advertising clicks all exhibit a significant upward trend. For example, total weekly revenue grew from approximately \$X million in the first week to \$Y million in the final week of the sample. %\ TODO: Fill in precise weekly revenue figures.

\subsubsection*{Model}

To estimate the average elasticity of vendor revenue with respect to clicks, we use a two-way fixed-effects model:

\begin{equation}
\log(\text{revenue}_{vt} + 1) = \alpha_v + \gamma_t + \beta \log(\text{clicks}_{vt} + 1) + \epsilon_{vt}
\end{equation}

where $\text{revenue}_{vt}$ is the revenue for vendor $v$ in week $t$, $\text{clicks}_{vt}$ is the number of clicks on vendor $v$'s ads, $\alpha_v$ is a vendor fixed effect, and $\gamma_t$ is a week fixed effect. The results are presented in Table \ref{tab:vendor_fe_results}.

\begin{table}[htbp!]
\centering
\caption{Fixed-Effects Model of Vendor-Week Revenue}
\label{tab:vendor_fe_results}
\begin{tabular}{lc}
\toprule
 & log(revenue + 1) \\
\midrule
log(clicks + 1) & 0.6422 \\
 & (0.0028) \\
\midrule
Vendor Fixed Effects & Yes \\
Week Fixed Effects & Yes \\
Observations & 979,290 \\
R-squared & 0.5574 \\
Within R-squared & 0.0734 \\
\bottomrule
\end{tabular}
\end{table}

Note: Standard errors are clustered at the vendor level. The coefficient for log(clicks + 1) is statistically significant at the p < 0.001 level.

The estimated elasticity of 0.6422 is statistically significant and suggests that, on average, a 1\% increase in clicks is associated with a 0.64\% increase in revenue. This result provides a baseline estimate of the average return to advertising across all vendors.

\subsection*{Vendor Heterogeneity}

We extend the analysis to account for vendor-level heterogeneity using a mixed-effects model with random slopes. The model is specified as:

\begin{equation}
\log(\text{revenue}_{vt} + 1) = \beta_v \log(\text{clicks}_{vt} + 1) + \alpha_v + \gamma_t + \epsilon_{vt}
\end{equation}

where $\beta_v$ is a vendor-specific click elasticity coefficient. We estimate these coefficients using an empirical Bayes approach that shrinks extreme estimates toward the global mean, reducing noise from vendors with limited data. The distribution of vendor-specific click elasticities reveals significant variation across vendors, with a mean elasticity of 0.77.

\begin{table}[htbp!]
\centering
\caption{Summary Statistics of Vendor Elasticities ($\beta_v$)}
\label{tab:vendor_beta_v_dist}
\begin{tabular}{lr}
\toprule
Statistic & Value \\
\midrule
Min. & 0.1942 \\
1st Qu. & 0.7487 \\
Median & 0.7788 \\
Mean & 0.7721 \\
3rd Qu. & 0.7995 \\
Max. & 1.2652 \\
\bottomrule
\end{tabular}
\end{table}

We use these vendor-specific elasticities to calculate the incremental Return on Ad Spend (iROAS) for each vendor, assuming a constant cost-per-click (CPC).

\subsubsection*{Incremental Return on Ad Spend (iROAS)}

To evaluate the profitability of advertising campaigns at a granular level, we utilize the Incremental Return on Ad Spend (iROAS). This metric quantifies the incremental revenue generated for each unit of advertising expenditure. iROAS is derived from vendor-specific click elasticity estimates and other key business metrics.

The iROAS is calculated as:
\begin{equation}
\text{iROAS} = \frac{\text{Vendor-Specific Click Elasticity} \times \text{Average Revenue per Click}}{\text{Assumed Cost per Click}}
\end{equation}

The components of the iROAS calculation are defined as follows: The vendor-specific click elasticity ($\beta_v$) is the elasticity coefficient estimated from the mixed-effects model, representing the percentage change in revenue for a one percent change in clicks for a specific vendor. The average revenue per click is the total revenue generated by a vendor divided by their total clicks, calculated as Total Revenue / Total Clicks. Finally, the assumed cost per click (CPC) is the average cost incurred for each click on an ad, which is an assumption based on platform-wide averages or specific campaign costs.

To illustrate the iROAS calculation, consider a representative vendor with median characteristics:

\begin{table}[htbp!]
\centering
\caption{Example iROAS Calculation Parameters}
\begin{tabular}{lr}
\toprule
Parameter & Value \\
\midrule
Vendor-specific elasticity ($\beta_v$) & 0.77 \\
Total revenue & \$10,000 \\
Total clicks & 500 \\
Average revenue per click & \$20 \\
Cost per click (marketplace rate) & \$0.75 \\
\bottomrule
\end{tabular}
\end{table}

Applying the iROAS formula yields:
\begin{equation}
\text{iROAS} = \frac{0.77 \times 20}{0.75} = 20.53
\end{equation}

This ratio indicates each advertising dollar generates \$20.53 in incremental revenue. Given the heterogeneity in vendor elasticities, we examine iROAS sensitivity to CPC levels:

\begin{table}[htbp!]
\centering
\caption{iROAS Sensitivity Analysis}
\begin{tabular}{lrrr}
\toprule
CPC Level & Min iROAS & Median iROAS & Max iROAS \\
& ($\beta_v$=0.19) & ($\beta_v$=0.78) & ($\beta_v$=1.27) \\
\midrule
\$0.50 (niche marketplace) & 7.77 & 31.15 & 50.61 \\
\$0.75 (typical marketplace) & 5.18 & 20.77 & 33.74 \\
\$1.16 (e-commerce average) & 3.35 & 13.43 & 21.81 \\
\$2.69 (Google Search average) & 1.44 & 5.79 & 9.41 \\
\bottomrule
\end{tabular}
\end{table}

The analysis reveals that marketplace advertising remains profitable across the elasticity distribution at typical marketplace CPCs (\$0.75), with only the bottom decile of vendors approaching break-even at e-commerce average rates.
