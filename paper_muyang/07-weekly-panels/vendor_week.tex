\subsection*{Staggered Adoption Analysis}

The preceding analysis relies on two-way fixed effects and mixed effects models, which impose restrictive assumptions on treatment effect homogeneity across time and units. To address concerns about bias from staggered treatment timing, we implement the estimator proposed by \citet{callaway2021difference}, which provides valid inference under heterogeneous treatment effects in settings where units adopt treatment at different times.

We define treatment as the first week a vendor wins any auction, representing the moment when their advertising becomes active on the platform. The panel comprises 846,430 observations from 142,920 vendors over 26 weeks. Of these vendors, 139,356 (97.5\%) eventually adopt advertising at some point during the sample period, while 3,564 (2.5\%) never win an auction and serve as the control group.\footnote{The high adoption rate reflects the nature of the marketplace where most active vendors participate in the advertising system. The never-treated group consists primarily of vendors who list products but never successfully compete in auctions.}

The Callaway-Sant'Anna estimator identifies group-time average treatment effects $ATT(g,t)$ for each cohort $g$ (defined by the period of first treatment) at each time $t$. These are then aggregated to obtain an overall average treatment effect on the treated (ATT) and event-study estimates $\theta(e)$ that trace out the treatment effect dynamics relative to the treatment onset.

Table \ref{tab:staggered_main} presents the main estimates using never-treated vendors as the control group.

\begin{table}[htbp!]
\centering
\caption{Staggered Difference-in-Differences Estimates}
\label{tab:staggered_main}
\begin{tabular}{lcccc}
\toprule
Outcome & ATT & SE & 95\% CI & Sig. \\
\midrule
Impressions & +1.061 & 0.021 & [1.020, 1.102] & *** \\
Clicks & +0.032 & 0.001 & [0.029, 0.034] & *** \\
Total GMV (\$) & +0.26 & 1.20 & [-2.09, +2.61] & \\
\bottomrule
\end{tabular}
\end{table}

Winning auctions increases weekly impressions by approximately 1.06 units and clicks by 0.032 units, both statistically significant at the 0.1\% level. The effect on vendor gross merchandise value (GMV), however, is not statistically distinguishable from zero. The point estimate of \$0.26 per vendor-week has a standard error of \$1.20, yielding a confidence interval that includes both positive and negative values.

The validity of difference-in-differences estimation requires parallel trends in the pre-treatment period. Table \ref{tab:staggered_pretrends} summarizes the pre-trends assessment.

\begin{table}[htbp!]
\centering
\caption{Pre-Trends Assessment}
\label{tab:staggered_pretrends}
\begin{tabular}{lccc}
\toprule
Outcome & Mean $\theta(e<0)$ & Significant Pre-periods & Joint $p$-value \\
\midrule
Impressions & +0.000048 & 0/22 & 0.615 \\
Clicks & 0.000000 & 0/22 & -- \\
Total GMV & 0.000000 & 0/22 & -- \\
\bottomrule
\end{tabular}
\end{table}

All three outcomes pass the parallel trends test. For impressions, the joint Wald test yields a $p$-value of 0.615, indicating no statistically significant deviation from parallel trends in the pre-treatment period. No individual pre-treatment coefficient achieves significance at the 5\% level.

Figure \ref{fig:event_study_impressions} displays the event study estimates for impressions. The pre-treatment coefficients cluster tightly around zero, providing visual confirmation of the parallel trends assumption. At the treatment onset ($e=0$), a sharp jump of approximately 0.83 impressions occurs. The effect persists and grows modestly over subsequent periods, reaching 1.21 impressions by twenty weeks post-treatment.

\begin{figure}[htbp!]
\centering
\begin{subfigure}[b]{0.32\textwidth}
\includegraphics[width=\textwidth]{figures/event_study_impressions.png}
\caption{Impressions}
\label{fig:event_study_impressions}
\end{subfigure}
\hfill
\begin{subfigure}[b]{0.32\textwidth}
\includegraphics[width=\textwidth]{figures/event_study_clicks.png}
\caption{Clicks}
\label{fig:event_study_clicks}
\end{subfigure}
\hfill
\begin{subfigure}[b]{0.32\textwidth}
\includegraphics[width=\textwidth]{figures/event_study_total_gmv.png}
\caption{Total GMV}
\label{fig:event_study_total_gmv}
\end{subfigure}
\caption{Event Study: Effect of Auction Wins on Vendor Outcomes}
\label{fig:event_studies}
\end{figure}

The results present a coherent picture of the advertising funnel. Winning auctions causally generates visibility (impressions) and engagement (clicks) for vendors. The click-through rate implied by the estimates is approximately 3.0\% (0.032/1.06), consistent with typical marketplace advertising benchmarks. However, the effect does not propagate to detectable increases in vendor sales within the same week. The null result for GMV may reflect several factors including organic purchases that would have occurred regardless, delayed conversion effects beyond the weekly window, or genuine ineffectiveness of marginal advertising. To validate this null finding and explore its boundaries, we conduct several further tests.

\subsection*{Robustness and Heterogeneity}

To validate the main findings and explore treatment effect heterogeneity, we perform three additional analyses. First, we test the sensitivity of the estimates to the choice of control group. Second, we investigate whether treatment effects differ for vendors with varying levels of pre-existing sales activity. Third, we compare our estimates to those from other modern difference-in-differences estimators.

\begin{table}[htbp!]
\centering
\caption{ATT Estimates with Alternative Control Group}
\label{tab:staggered_robustness_controls}
% \ TODO: Run the Callaway & Sant'Anna estimation using the "not-yet-treated" as the control group and report the ATT estimates for the three main outcomes here.
\begin{tabular}{lcc}
\toprule
& (1) & (2) \\
Outcome & Never-Treated (Main) & Not-Yet-Treated \\
\midrule
Impressions & +1.061 & [Coefficient] \\
& (0.021) & [Std. Error] \\
Clicks & +0.032 & [Coefficient] \\
& (0.001) & [Std. Error] \\
Total GMV (\$) & +0.26 & [Coefficient] \\
& (1.20) & [Std. Error] \\
\bottomrule
\end{tabular}
\end{table}

\begin{table}[htbp!]
\centering
\caption{Heterogeneity of GMV Effect by Pre-Treatment Sales}
\label{tab:staggered_heterogeneity}
% \ TODO: Split the sample of treated vendors into terciles based on their average weekly GMV in the four weeks prior to treatment. Run the C&S estimation for each subgroup and report the ATT on Total GMV.
\begin{tabular}{lccc}
\toprule
& (1) & (2) & (3) \\
& Low Tercile & Middle Tercile & High Tercile \\
\midrule
ATT on Total GMV (\$) & [Coefficient] & [Coefficient] & [Coefficient] \\
& ([Std. Error]) & ([Std. Error]) & ([Std. Error]) \\
\bottomrule
\end{tabular}
\end{table}

\begin{table}[htbp!]
\centering
\caption{ATT Estimates with Alternative Staggered DiD Estimators}
\label{tab:staggered_robustness_estimators}
% \ TODO: Estimate the ATT on Total GMV using the estimators from Sun & Abraham (2021) and Borusyak et al. (2021) and compare them to the C&S result.
\begin{tabular}{lccc}
\toprule
& (1) & (2) & (3) \\
& C\&S (Main) & Sun \& Abraham & Borusyak et al. \\
\midrule
ATT on Total GMV (\$) & +0.26 & [Coefficient] & [Coefficient] \\
& (1.20) & [Std. Error] & [Std. Error] \\
\bottomrule
\end{tabular}
\end{table}

% [Removed stubbed subsection "Further Considerations" on 2026-02-05]
