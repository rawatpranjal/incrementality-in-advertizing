\documentclass{article}
\usepackage{amsmath}
\usepackage{amsfonts}
\usepackage{booktabs}
\usepackage{float}
\usepackage[margin=1in]{geometry}
\usepackage{setspace}
\usepackage[skip=8pt plus1pt, indent=0pt]{parskip}
\usepackage{microtype}
\raggedbottom

\begin{document}

\section*{Model 5: The Dynamic Causal Model}
\vspace{-0.5em}
\subsection*{Objective}
The objective is to synthesize the ad-stock formulation (Model 3) with the instrumental variable strategy (Model 4) to create a single, comprehensive model. This final model is designed to estimate the time-decaying, heterogeneous effects of advertising in a causally valid manner using real-world, non-experimental observational data. It simultaneously solves the problems of static time assumptions and endogeneity from ad targeting.

\subsection*{The Model Formulation}
\vspace{-0.5em}
The underlying structure of the model is the same as the dynamic ad-stock model (Model 3). It describes the instantaneous conversion rate as a function of the decaying influence of past ad exposures.

\vspace{-0.5em}
\[ y_i(t) = \alpha(t|W) + \sum_{k} \beta_k x_{ik}(t) + \varepsilon_i(t) \]
\vspace{-0.5em}

The crucial difference is not the equation itself, but the method used to estimate the $\beta_k$ coefficients. In this final model, we recognize that the realized ad stock, $x_{ik}(t)$, is endogenous and requires an instrumental variable for unbiased estimation.

\subsection*{The Dynamic Instrumental Variable Framework}
\vspace{-0.5em}
We extend the IV logic from Model 4 into the continuous-time domain. The instrument must also be a dynamic, time-varying variable: the "ghost ad stock."

\begin{table}[htbp]
\centering
\caption{Dynamic Endogenous Treatment and Instrument}
\begin{tabular}{@{}p{4cm}p{8cm}@{}}
\toprule
Variable & Definition and Role \\
\midrule
\textbf{Endogenous Treatment:} \newline Realized Ad Stock ($x_{ik}(t)$) & The ad stock for characteristic $k$ at time $t$, calculated by summing the decaying influence of all \textbf{actual, won impressions}. This variable is endogenous because winning an impression is correlated with user intent. \\
\addlinespace[0.5em]
\textbf{Instrument:} \newline Ghost Ad Stock ($z_{ik}(t)$) & The ad stock for characteristic $k$ at time $t$ that \textbf{would have been generated} by our randomized "intent-to-bid" mechanism. It is calculated by summing the decaying influence of all \textbf{ghost ads}. By construction, this variable is correlated with the realized ad stock but is uncorrelated with the unobserved user intent, $\varepsilon_i(t)$. \\
\bottomrule
\end{tabular}
\end{table}

\subsection*{The Dynamic 2SLS Estimation Process}
\vspace{-0.5em}
The model is estimated using a two-stage process on the continuous-time data, made tractable by the positive/negative sampling scheme described in Model 3.

\begin{table}[htbp]
\centering
\caption{The Dynamic 2SLS Estimation Process}
\begin{tabular}{@{}lp{8.5cm}@{}}
\toprule
Stage & Interpretation \\
\midrule
\textbf{First Stage} & The endogenous realized ad stock, $x_{ik}(t)$, is regressed on the instrumental ghost ad stock, $z_{ik}(t)$, and other exogenous controls. This stage effectively filters the realized ad stock, producing a predicted version, $\hat{x}_{ik}(t)$, that is driven only by the exogenous randomization. \\
\addlinespace[0.5em]
\textbf{Second Stage} & The conversion outcome, $y_i(t)$, is regressed on the predicted ad stock, $\hat{x}_{ik}(t)$, from the first stage. The estimation is performed using the weighted sample of positive and negative moments. The resulting coefficients, $\hat{\beta}_{\text{IV}, k}$, are consistent, causally valid estimates of the incremental effect of a unit of ad stock for each characteristic $k$. \\
\bottomrule
\end{tabular}
\end{table}

\subsection*{Implications of the Final Model}
\vspace{-0.5em}
This synthesized model represents a complete solution, enabling a sophisticated, data-driven advertising system.

\begin{table}[htbp]
\centering
\caption{Capabilities of the Dynamic Causal Model}
\begin{tabular}{@{}lp{8.5cm}@{}}
\toprule
Capability & Description \\
\midrule
\textbf{Causal and Dynamic Attribution} & When a conversion occurs at time $t_c$, credit is attributed to past impressions based on their remaining \textit{causal} influence at that moment. The model correctly accounts for both time decay and the true incremental effect, purged of selection bias. \\
\addlinespace[0.5em]
\textbf{Optimal Bidding} & The bidding rule is based on the integral of the future \textit{causal} effect of an impression. The maximum bid for an impression with characteristics $\mathbf{x}_{ij}$ is proportional to $\boldsymbol{\beta}_{\text{IV}}^\top \mathbf{x}_{ij}$, where $\boldsymbol{\beta}_{\text{IV}}$ is the vector of causally-identified coefficients. This ensures that bids reflect true incremental value, not correlational noise. \\
\addlinespace[0.5em]
\textbf{Accurate Forecasting and Budgeting} & The model can be used to simulate the impact of different budget scenarios or targeting strategies. Because the parameters are causal, the forecasts are robust to changes in targeting strategy, unlike correlational models which would break down. \\
\bottomrule
\end{tabular}
\end{table}

In summary, Model 5 combines the realistic time dynamics of the ad-stock formulation with the rigorous causal guarantees of the instrumental variable approach. It provides a theoretically sound and practically implementable framework for measuring and optimizing advertising incrementality in complex, real-world environments.

\end{document}