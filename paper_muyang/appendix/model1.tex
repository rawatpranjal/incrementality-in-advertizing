\documentclass{article}
\usepackage{amsmath}
\usepackage{amsfonts}
\usepackage{booktabs}
\usepackage{float}
\usepackage[margin=1in]{geometry}

\begin{document}

\section*{Model 1: Simple Model of Incrementality and Attribution}

Here we discuss simple models of incrementality from the viewpoint of a single vendor (brand, advertizer) running digital advertising campaigns. The model facilitates the separation of organic outcomes from those induced by advertising. The simplest possible linear model decomposes a user-level outcome into a baseline component, an advertising treatment effect, and a random error term:

\[ y_i = \alpha + \beta x_i + \varepsilon_i \]


\begin{table}[H]
\centering
\caption{Model Components and Interpretation}
\begin{tabular}{@{}lp{10cm}@{}}
\toprule
Component & Interpretation \\
\midrule
$i$ & The unit of analysis is the user $i$. With changes to privacy, the user may not persist accross multiple sessions. \\
$y_i$ & The observed outcome for user $i$. This is typically a binary event (e.g., conversion=1, no conversion=0) aggregated over a time period. \\
$\alpha$ & The baseline outcome rate. It represents the expected outcome for a user who has zero ad exposure ($x_i=0$). Industry benchmarks show average landing-page conversion rates of approximately 2.35\%, making baselines of 1-3\% plausible. \\
$\beta$ & The incremental effect. It quantifies the average causal effect of one additional unit of ad exposure on the outcome. \\
$x_i$ & The measure of ad exposure for user $i$. In the context of real-time bidding (RTB), this is a count of ad impressions. \\
$\varepsilon_i$ & The stochastic error term. It captures all unobserved factors that influence the outcome $y_i$ but are not included in the model. \\
\bottomrule
\end{tabular}
\end{table}

Here are some examples of real world settings that influence the parameters of the model.

\begin{table}[H]
\centering
\caption{Factors Influencing Model Parameters}
\begin{tabular}{@{}p{2.5cm}p{6cm}p{6cm}@{}}
\toprule
Parameter & Scenario & Plausible Value / State \\
\midrule
$\alpha$ (Baseline) & High-intent user (e.g., searched "buy iPhone 14") & High $\alpha$. User is likely to convert regardless of ads. \\
& Low-intent user (e.g., browsing news) & Low $\alpha$. Conversion is unlikely without a strong stimulus. \\
& Strong brand with loyal customers & High $\alpha$. Organic traffic and direct conversions are frequent. \\
\addlinespace
$\beta$ (Incrementality) & Compelling call-to-action (e.g., "50\% off for 24 hours") & High $\beta$. The ad provides new information that drives action. \\
& Brand awareness campaign for a mature product & Low or near-zero $\beta$. The ad reinforces existing knowledge. \\
& Retargeting a user who abandoned a full shopping cart & High $\beta$. The ad serves as a timely reminder to a high-intent user. Retargeting CTRs ($\sim$0.7\%) are typically 10$\times$ higher than generic display ads. \\
\addlinespace
Plausible Range for $\beta$ & A single ad impression rarely causes a sale. & Typically very small, e.g., $10^{-5}$ to $10^{-3}$ (0.001\% to 0.1\%). This aligns with typical display CTRs ($\sim$0.1\%) and conversion rates ($\sim$2.35\%), yielding conversion probabilities per impression of $\sim$2$\times$10$^{-5}$. \\
\addlinespace
$\varepsilon_i$ (Unobserved) & Competitor launches a major sale. & A negative shock to $\varepsilon_i$ for some users. \\
& User receives a recommendation from a friend. & A positive shock to $\varepsilon_i$. \\
& User is exposed to an offline ad (e.g., TV, billboard). & A positive, unobserved influence captured in $\varepsilon_i$. \\
\bottomrule
\end{tabular}
\end{table}

\subsection*{Ad Exposures $x_i$}
The definition of the treatment variable $x_i$ is determined by the advertising channel and the point of intervention.

\begin{table}[H]
\centering
\caption{Definition and State of the Treatment Variable $x_i$}
\begin{tabular}{@{}lp{10cm}@{}}
\toprule
Concept & Description \\
\midrule
Impressions vs. Clicks & In display advertising and RTB, the intervention is showing an ad. Therefore, $x_i$ counts impressions. The advertiser pays per impression (CPM). If the model were for paid search, where payment is per click (CPC), $x_i$ could be defined as clicks, and $\beta$ would measure conversions per click. \\
\addlinespace
$x_i = 0$ & Could represents a user in the control group. This user was either intentionally withheld from advertising (in an experiment) or was simply not reached by the campaign. Their expected outcome is $E[y_i|x_i=0] = \alpha$. \\
\addlinespace
$x_i > 0$ & Represents a user in the treatment group who was exposed to one or more ad impressions. Their expected outcome is $E[y_i|x_i>0] = \alpha + \beta x_i$. \\
\bottomrule
\end{tabular}
\end{table}

\subsection*{Breakdown in Assumptions}
The model's validity rests on key assumptions that are often violated in practice, primarily through endogeneity and spillovers (interference).

\begin{table}[H]
\centering
\caption{Common Violations of Model Assumptions}
\begin{tabular}{@{}lp{10cm}@{}}
\toprule
Violation & Mechanism and Implication \\
\midrule
Endogeneity & This occurs when an unmeasured factor in $\varepsilon_i$ influences both ad exposure $x_i$ and the likelihood of conversion $y_i$. Randomized experiments remain the primary solution to eliminate this bias. \\
\addlinespace
\textit{Targeting} & Advertisers show more ads to users they believe are likely to convert. This high purchase intent is an unobserved component of $\varepsilon_i$. Consequently, users with high $\varepsilon_i$ also have high $x_i$, creating a positive correlation that leads to an upwardly biased $\hat{\beta}$. \\
\addlinespace
\textit{Activity Bias} & Users who are more active online are naturally exposed to more ads (high $x_i$) and may also be more likely to transact online for unobserved reasons (high $\varepsilon_i$). This again induces a positive correlation and an overestimated $\hat{\beta}$. \\
\addlinespace
Spillovers (SUTVA Violation) & The treatment of one user affects the outcome of another, meaning observations are not independent. This interference can invalidate user-level models. Field experiments have shown spillover effects up to 5$\times$ larger than direct advertiser gains. \\
\addlinespace
\textit{Word of Mouth} & User 1 sees an ad ($x_1>0$), converts, and tells User 2 about the product. User 2, who saw no ad ($x_2=0$), then converts. The treatment applied to User 1 affects the outcome of User 2. \\
\addlinespace
\textit{General Equilibrium} & A successful ad campaign for a limited-stock item causes treated users to purchase it. When untreated users later try to buy the item, it is out of stock. The treatment for one group negatively affects the outcome for another. \\
\bottomrule
\end{tabular}
\end{table}
\subsection*{Profitability}

The model must incorporate the financial context of the advertiser. This is achieved by introducing three key variables: the value of a conversion, the profit margin, and the cost of ad exposure.

\begin{table}[H]
\centering
\caption{Economic Components for Bidding}
\begin{tabular}{@{}lp{6cm}p{5cm}@{}}
\toprule
Component & Interpretation & Unit \\
\midrule
$v_i$ & The total value or revenue generated from a conversion event for user $i$. This can be a single transaction value or a customer lifetime value (LTV). & Currency per Conversion \\
$m_i$ & The gross profit margin for a conversion from user $i$. This is the fraction of revenue that is profit before accounting for marketing spend. & Dimensionless (e.g., 0.4 for 40\%) \\
$c$ & The cost to the advertiser for a single ad exposure (impression). & Currency per Exposure \\
\bottomrule
\end{tabular}
\end{table}

An ad exposure is a profitable investment only if the incremental gross profit it generates exceeds its cost. The expected incremental gross profit from one ad exposure is the product of the incremental conversion probability ($\beta$), the conversion value ($v_i$), and the gross margin ($m_i$). The profitability condition is therefore:

\[ \beta \cdot m_i \cdot v_i > c \]

This condition directly yields a bidding rule. In a second-price auction, the optimal strategy is to bid one's true valuation. The maximum bid, or willingness-to-pay, for an impression is the expected incremental gross profit it generates.

\[ \text{Maximum Bid} = \beta \cdot m_i \cdot v_i \]

\subsection*{The Cost Per Incremental Action (CPIA)}

The profitability condition can be rearranged to define a key performance metric: the Cost Per Incremental Action (CPIA). This metric represents the cost of acquiring one additional conversion that would not have happened otherwise.

\[ \text{CPIA} = \frac{c}{\beta} \]

The investment is profitable if the cost to generate an incremental action is less than the gross profit derived from that action.

\[ \text{CPIA} < m_i \cdot v_i \]

This formulation allows for direct comparison of advertising efficiency across different channels and tactics, normalizing for varying costs ($c$) and effectiveness ($\beta$).

\subsection*{Cost-Per-Click Business Model}
The objective is to adapt the incrementality framework to a CPC pricing model, prevalent in paid search advertising. The problem requires re-specifying the model's treatment variable and re-interpreting its parameters to align with the unit of purchase, which is the click rather than the impression. The core linear structure remains, but its components are mapped to different events in the user journey.

\begin{table}[H]
\centering
\caption{Model Re-specification for CPC Context}
\begin{tabular}{@{}lp{6cm}p{5cm}@{}}
\toprule
Component & CPC Model Interpretation & Unit \\
\midrule
$y_i$ & The conversion outcome for user $i$. & Conversions per User \\
$x_i$ & The number of paid ad clicks by user $i$. & Clicks per User \\
$\beta_{cpc}$ & The incremental conversion rate per click. It measures the additional probability of conversion given a paid click, isolating the ad's causal influence from the user's inherent intent signaled by the click. & Conversions per Click \\
$\alpha$ & The baseline conversion rate for users who do not click on a paid ad. This captures conversions from organic search, direct traffic, or other channels. & Conversions per User \\
$c_{cpc}$ & The cost to the advertiser for a single click. Industry average CPCs are \$2.69 for search and \$0.63 for display. & Currency per Click \\
\bottomrule
\end{tabular}
\end{table}

The profitability condition and bidding rule are reformulated around the click event. The investment is profitable if the incremental gross profit from a click exceeds the cost of that click.

\[ \beta_{cpc} \cdot m_i \cdot v_i > c_{cpc} \]

This directly informs the maximum price an advertiser should bid for a click in an auction.

\[ \text{Maximum CPC Bid} = \beta_{cpc} \cdot m_i \cdot v_i \]

\subsection*{ROAS vs. Incremental ROAS (iROAS)}
The objective is to define Return on Ad Spend (ROAS) and Incremental ROAS (iROAS) within the model's framework. Standard ROAS is a correlational metric that is inflated by baseline conversions ($\alpha$), while iROAS isolates the causal return from advertising ($\beta$).

\begin{table}[H]
\centering
\caption{Formal Definitions and Interpretation}
\begin{tabular}{@{}lp{5.5cm}p{5.5cm}@{}}
\toprule
Metric & Formula & Interpretation \\
\midrule
ROAS & $\frac{(\alpha + \beta x_i) \cdot v_i}{x_i \cdot c}$ & Correlational. Total observed revenue divided by ad spend. Incorrectly attributes baseline revenue to ads. \\
\addlinespace
iROAS & $\frac{\beta \cdot v_i}{c}$ & Causal. Revenue generated by ad spend. The correct metric for evaluating marginal ad profitability. \\
\bottomrule
\end{tabular}
\end{table}


\subsection*{Numerical Example}

An online software company runs a paid search campaign to acquire new subscribers. The key business metrics are:

\begin{table}[H]
\centering
\caption{Economic and Business Parameters}
\begin{tabular}{@{}lp{10cm}@{}}
\toprule
Parameter & Value and Context \\
\midrule
Conversion Value ($v$) & \$500 (Customer Lifetime Value of a new subscriber) \\
Gross Margin ($m$) & 80\% (Typical for software products) \\
Cost per Click ($c_{cpc}$) & \$2.50 (Market rate for core keywords) \\
\bottomrule
\end{tabular}
\end{table}

For this simulation, the true underlying data generating process is defined by $\alpha = 0.01$ (baseline conversion rate from non-paid channels) and $\beta_{cpc} = 0.05$ (the true incremental conversion lift from a paid click). Average paid-search conversion rates are approximately 3.75\%, making a 6\% total rate plausible for high-intent keywords. A sample of the firm's data is presented below.

\begin{table}[H]
\centering
\caption{Sample of Synthetic User-Level Data for Paid Search}
\begin{tabular}{@{}ccc@{}}
\toprule
User ID & Paid Clicks ($x_i$) & Converted ($y_i$) \\
\midrule
1 & 0 & 0 \\
2 & 1 & 1 \\
3 & 2 & 0 \\
4 & 0 & 0 \\
5 & 1 & 0 \\
... & ... & ... \\
\bottomrule
\end{tabular}
\end{table}

\subsubsection*{Model Estimation}
An OLS regression of $y_i$ on $x_i$ and an intercept is performed on the full dataset (N=500,000 users). The analysis assumes the data originates from a randomized experiment, ensuring the OLS estimates are unbiased. The regression yields the following estimated coefficients:

\begin{table}[H]
\centering
\caption{Estimated Model Parameters}
\begin{tabular}{@{}lc@{}}
\toprule
Parameter & Estimated Value \\
\midrule
Estimated Baseline ($\hat{\alpha}$) & 0.0098 \\
Estimated Incremental Effect per Click ($\hat{\beta}_{cpc}$) & 0.051 \\
\bottomrule
\end{tabular}
\end{table}

The resulting estimated model is: $E[y_i] = 0.0098 + 0.051 \cdot x_i$. All subsequent business decisions are based on these estimates.

\subsubsection*{Profitability}
The estimated parameter $\hat{\beta}_{cpc}$ is used to evaluate campaign profitability and establish a valuation-based bidding rule.

\begin{table}[H]
\centering
\caption{Profitability Analysis Based on Estimated Model}
\begin{tabular}{@{}lp{10cm}@{}}
\toprule
Analysis & Calculation and Interpretation \\
\midrule
Maximum CPC Bid & The maximum willingness-to-pay for one click is the expected incremental gross profit. \\
($\hat{\beta}_{cpc} \cdot m \cdot v$) & \textit{Calculation:} $0.051 \cdot 0.80 \cdot \$500 = \$20.40$ \\
& \textit{Interpretation:} The firm should bid no more than \$20.40 for a click on these keywords. Since the current cost is \$2.50, the clicks are highly profitable. \\
\addlinespace
Incremental CPA (iCPA) & The cost to acquire one truly incremental conversion via paid search. \\
($c_{cpc} / \hat{\beta}_{cpc}$) & \textit{Calculation:} $\$2.50 / 0.051 \approx \$49.02$ \\
& \textit{Interpretation:} It costs the firm \$49.02 to generate one new subscriber who would not have signed up otherwise. \\
\addlinespace
Profitability Check & The iCPA is compared to the gross profit from a single conversion ($m \cdot v$). \\
($\text{iCPA} < m \cdot v$) & \textit{Calculation:} Gross Profit = $0.80 \cdot \$500 = \$400.00$. \\
& \textit{Conclusion:} Since \$49.02 (iCPA) is substantially less than \$400.00 (Gross Profit), the campaign is highly profitable. \\
\bottomrule
\end{tabular}
\end{table}

\subsubsection*{Counterfactuals}
The estimated model enables the firm to evaluate strategic decisions and quantify past performance.

\begin{table}[H]
\centering
\caption{Counterfactual Scenarios for Paid Search}
\begin{tabular}{@{}lp{10cm}@{}}
\toprule
Scenario & Question and Model-Driven Answer \\
\midrule
1. Budget Impact & \textbf{Question:} "Last quarter, we purchased 10,000 clicks at \$2.50 each. What was the net incremental profit from this investment?" \\
& \textbf{Answer:} \\
& 1. Incremental Conversions = $10,000 \cdot \hat{\beta}_{cpc} = 10,000 \cdot 0.051 = 510$. \\
& 2. Incremental Gross Profit = $510 \cdot (\$500 \cdot 0.80) = 510 \cdot \$400 = \$204,000$. \\
& 3. Total Ad Cost = $10,000 \cdot \$2.50 = \$25,000$. \\
& 4. Net Incremental Profit = $\$204,000 - \$25,000 = \$179,000$. \\
& The model estimates the investment generated \$179,000 in incremental profit. \\
\addlinespace
2. Keyword Expansion & \textbf{Question:} "We are considering expanding to broader, top-of-funnel keywords. The market CPC is lower at \$1.50, but we estimate the incremental effect will also be lower, with an expected $\beta_{cpc}$ of 0.03. Is this a profitable expansion?" \\
& \textbf{Answer:} \\
& First, calculate the maximum bid (willingness-to-pay) for clicks in this new segment. \\
& Max CPC Bid (New Segment) = $0.03 \cdot 0.80 \cdot \$500 = \$12.00$. \\
& The market price (\$1.50) is well below the firm's maximum valuation (\$12.00). The expansion is a profitable opportunity and should be pursued. \\
\bottomrule
\end{tabular}
\end{table}

\end{document}