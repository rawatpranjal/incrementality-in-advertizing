
\section*{Advertising in a Social Marketplace}

The marketplace that we study is a consumer-to-consumer (C2C) online marketplace that facilitates the sale of secondhand goods, primarily apparel, alongside home goods and electronics. The business model is commission-based; the firm collects a fee for each transaction. For sales of \$15 or more, the marketplace claims a 20\% commission on the final sale price; for transactions below this threshold, a flat \$2.95 fee applies. This structure aligns marketplace revenue with the gross merchandise value (GMV) transacted between users.

The marketplace supports two distinct purchase mechanisms. Buyers may either accept a seller's listed price through an immediate ``Buy Now'' transaction, or initiate a binding negotiation via the ``Make an Offer'' feature. In the latter case, sellers can accept the proposed price, counter with an alternative, or decline; if accepted, the transaction proceeds at the negotiated amount.

Because of this dual-path structure, sellers often list items above their true reservation price to create negotiation headroom, a practice reinforced by marketplace features such as ``Offer to Likers,'' which permits sellers to extend private discounts to users who have favorited a listing, subject to a minimum 10\% price reduction from the listed value. These pricing dynamics introduce strategic considerations into the interpretation of advertising effects, as promoted listings may carry artificially inflated sticker prices designed to anchor subsequent bargaining rather than signal true willingness-to-accept.

The marketplace integrates social networking features into the e-commerce experience. User actions such as following, liking, and sharing listings influence visibility. Listing prominence is earned through social participation, not purchased directly. The marketplace's default search algorithm prioritizes recently shared items, making frequent seller activity a primary driver of organic reach independent of item relevance or price. Features such as themed virtual shopping parties and live-streaming showcases further integrate social features into transactions.

A corporate acquisition in January 2023 changed the marketplace's strategy. The new ownership, facing persistent unprofitability, prioritized higher-margin monetization channels beyond the commission-based model, which had accounted for approximately 96\% of total revenue in 2021.

This led to the introduction of an advertising tool for sellers in the United States, creating a dual revenue model. The service is an automated system where sellers set a weekly budget to have their listings promoted in high-visibility placements, such as search results, brand pages, and chronological ``Just In'' feeds that showcase newly listed items. The system operates on a closet-wide basis; sellers cannot manually select or exclude specific listings from promotion, with the algorithm automatically determining which items to surface from the seller's entire available inventory. It operates on a cost-per-click (CPC) basis, with variable pricing and weekly billing cycles. Promoted listings carry a distinctive visual treatment: each sponsored item displays either a ``Promoted'' label or an ``ad by [Seller Name]'' designation, so users can distinguish paid placements from organic results.

The ad platform's algorithm, built on machine learning models trained on the marketplace's organic sales data and developed in partnership with an ad platform provider, determines which listings to surface and when, with the stated objective of maximizing conversions. A seller's budget size directly impacts listing visibility; the ad platform provides recommended budgets based on inventory size and advises increases when sellers add new items. Budget adjustments take effect at the start of the next weekly cycle, and the tool is available to all sellers without minimum participation requirements. Apart from setting the weekly budget, sellers possess no granular controls: the system does not permit manual bidding on keywords or daily spending caps, rendering sellers price-takers in the ad auction.

The system employs what the marketplace terms ``halo attribution''—a 28-day window in which any purchase from a seller's inventory is credited to the advertising campaign if the buyer previously clicked on any of that seller's promoted listings. Consider a scenario where a user clicks on a promoted handbag listing but ultimately purchases a pair of shoes from the same seller's inventory three weeks later; under this attribution logic, both the initial click and the eventual shoe purchase would be credited to the advertising campaign. The seller dashboard provides metrics such as Return on Ad Spend (ROAS), total attributed sales, and ranks items by promotional performance. To drive adoption, the marketplace offers free trial periods that automatically convert to paid weekly subscriptions. The advertising tool allows sellers to buy visibility, unlike the original social engagement model.

The marketplace's inventory consists of over 52 million unique listings, reflecting its C2C model where each item is a distinct entity. The catalog is predominantly composed of apparel, accessories, and footwear, with smaller segments for home goods and media. Listings are highly descriptive, containing structured metadata such as brand, color, size, and style tags, alongside unstructured seller-provided text. The median listing price is \$25.00, though the distribution is right-skewed by high-value luxury goods. The vast majority of items fall within the \$10 to \$50 price range.

Popular categories are defined by well-known mid-market and high-end brands, as well as broad style attributes. Most items are unique with no purchase history, creating data sparsity for ad prediction models that rely on historical performance to estimate conversion probability.