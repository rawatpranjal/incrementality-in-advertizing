\subsection{Micro-Level Funnel Analysis}

While aggregated time-series models (Section 2) reveal ambiguous relationships, a granular micro-level analysis is required to isolate the causal effect of advertising from confounding factors. To this end, we construct a user-journey panel to facilitate a more granular causal analysis of the advertising funnel. This micro-level approach allows us to control for a rich set of confounding variables and explore the nuanced, heterogeneous effects of advertising on user behavior.

\subsubsection{Baseline Model of Direct Conversion}

To estimate the direct causal effect of clicks on purchase conversion, we construct a panel dataset where the unit of analysis is the (product, journey) pair. A user journey is defined as a sequence of user actions delimited by a period of inactivity of two hours. Each observation $(i, j)$ represents a unique product $i$ that received at least one ad impression within a unique user journey $j$. Table \ref{tab:summary_stats_revised} presents the descriptive statistics for this analytical sample.

\begin{table}[h!]
\centering
\caption{Descriptive Statistics of the User-Journey Panel}
\label{tab:summary_stats_revised}
\begin{tabular}{lrrrrr}
\hline\hline
\textbf{Variable} & \textbf{Mean} & \textbf{Median} & \textbf{Std. Dev.} & \textbf{Min} & \textbf{Max} \\
\hline
\textit{Outcomes} & & & & & \\
\quad Purchase (binary) & 0.00017 & 0 & 0.013 & 0 & 1 \\
\quad Revenue (\$) & 0.028 & 0 & 2.14 & 0 & 1200 \\
\textit{Treatment} & & & & & \\
\quad Clicks (count) & 0.041 & 0 & 0.22 & 0 & 6 \\
\quad Was Clicked (binary) & 0.036 & 0 & 0.187 & 0 & 1 \\
\textit{Controls} & & & & & \\
\quad Price (\$) & 166.34 & 40.00 & 833.63 & 3.00 & 80,081 \\
\hline
\multicolumn{6}{p{0.95\textwidth}}{\footnotesize \textit{Notes:} N = 269,276 product-journey observations, constructed from 1,124 unique users across 7,820 journeys. The unit of observation is a unique product impressed within a unique user journey. Median is included for skewed variables like Price and Revenue.}
\end{tabular}
\end{table}

On this data, we fit two baseline models: a logistic regression on the purchase outcome and an OLS regression on log-transformed revenue. The models are specified as:

$ \text{logit}(P(\text{Purchase}_{ij} = 1)) = \beta_0 + \beta_1 \cdot \text{Clicks}_{ij} + \mathbf{X}_{ij}'\gamma + \epsilon_{ij} \quad (1) $
$ \ln(\text{Revenue}_{ij} + 1) = \beta_0 + \beta_1 \cdot \text{Clicks}_{ij} + \mathbf{X}_{ij}'\gamma + \epsilon_{ij} \quad (2) $

Table \ref{tab:baseline_models_revised} presents the results. The coefficient on Clicks on Product is positive and highly significant in both models. The logistic regression yields a coefficient of 2.225, which corresponds to an odds ratio of 9.25 ($\exp(2.225)$), indicating that each additional click on a product increases the odds of it being purchased by a factor of 9.25. The OLS model suggests that each click is associated with a 1.5\% increase in expected revenue.

\begin{table}[h!]
\centering
\caption{Baseline Regression Models of Direct Conversion}
\label{tab:baseline_models_revised}
\begin{tabular}{lcc}
\hline
\hline
 & (1) & (2) \\
 & Purchase (Logit) & log(Revenue + 1) \\
\hline
Clicks on Product & 2.225*** & 0.015*** \\
 & (0.151) & (0.003) \\
Log of Price & -0.495*** & -0.0002*** \\
 & (0.190) & (0.00007) \\
\hline
Controls & & \\
\quad Product Characteristics & Yes & Yes \\
\quad Journey Context & Yes & Yes \\
\quad User History & Yes & Yes \\
\hline
Observations & 269,276 & 269,276 \\
Pseudo R-squared & 0.246 & -- \\
Adj. R-squared & -- & 0.006 \\
\hline
\hline
\multicolumn{3}{p{0.9\textwidth}}{\footnotesize \textit{Notes:} Standard errors clustered at the user level in parentheses. The dependent variable for the Logit model is a binary indicator for purchase. Both models include the full set of control variables, including product, journey, and user history characteristics. *** p<0.01, ** p<0.05, * p<0.1}
\end{tabular}
\end{table}

\subsubsection{Robustness and Specification Checks}

To ensure the robustness of our baseline finding, we conduct two specification checks. First, we test for non-linearities in the click effect. Second, we examine the sensitivity of our estimates to the inclusion of different sets of control variables.

To test for diminishing returns, we introduce a quadratic term, $\text{Clicks}_{ij}^2$, into the logistic specification. The results confirm a non-linear relationship. As shown in Figure \ref{fig:diminishing_returns}, the predicted probability of purchase increases steeply with the first click, continues to rise at a decreasing rate for the second and third clicks, and then flattens. This provides strong visual evidence of diminishing returns to clicks.

\begin{figure}[h!]
\centering
% In your actual document, you would generate this plot and save it as a file.
% \includegraphics[width=0.8\textwidth]{figures/diminishing_returns_plot.png}
\fbox{Placeholder for Diminishing Returns Plot}
\caption{Predicted Probability of Purchase by Number of Clicks}
\label{fig:diminishing_returns}
{\footnotesize \textit{Notes:} This figure plots the predicted probability of purchase from the logistic regression model including a quadratic term for clicks. All other control variables are held at their mean values. The shaded area represents the 95\% confidence interval.}
\end{figure}

Furthermore, the main effect of clicks remains stable and highly significant across various model specifications, including models with only base controls, and models with additional historical and competitive context controls. This suggests that our baseline finding is not an artifact of a specific model choice.

\subsubsection{Heterogeneity of the Click Effect}

Building on prior work that highlights the importance of consumer heterogeneity (e.g., Blake et al., 2015), we investigate whether the click effect varies across several key dimensions. Figure \ref{fig:heterogeneity_plot} presents the estimated odds ratios for clicks across different subgroups.

\begin{figure}[h!]
\centering
% In your actual document, you would generate this plot and save it as a file.
% \includegraphics[width=0.8\textwidth]{figures/heterogeneity_plot.png}
\fbox{Placeholder for Heterogeneity Coefficient Plot}
\caption{Heterogeneity of the Click Effect on Purchase Probability}
\label{fig:heterogeneity_plot}
{\footnotesize \textit{Notes:} This figure displays the odds ratio (and 95\% confidence interval) for the effect of `Clicks on Product` from separate logistic regressions run on each subgroup. The vertical dashed line indicates an odds ratio of 1 (no effect).}
\end{figure}

The results reveal substantial variation. The effect is nearly twice as large for users with a low history of purchasing and is stronger in shorter, more focused journeys. Notably, the effect is most pronounced for products in the medium-to-high price range (\$41-\$75). These findings suggest that advertising is most effective as a persuasion tool for new or undecided users undertaking a considered purchase.

\subsubsection{Causal Mechanisms: Timing and Choice}

To deepen our causal understanding, we employ two advanced modeling frameworks that move beyond a simple binary outcome to explain the underlying mechanisms of how clicks work.

First, we use a Cox Proportional Hazards model to analyze the effect of clicks on the *timing* of a purchase. The model is specified as:

$ h(t | \mathbf{Z}_j) = h_0(t) \exp(\beta_1 \text{TotalClicks}_j + \mathbf{Z}_j'\gamma) \quad (3) $

where $h(t | \mathbf{Z}_j)$ is the hazard rate at time $t$ for journey $j$, $h_0(t)$ is the baseline hazard, and $\mathbf{Z}_j$ is a vector of journey-level covariates. The estimated hazard ratio for $\text{TotalClicks}_j$ is 1.017 (p < 0.01), implying that each additional click in a journey increases the instantaneous rate of purchase by 1.7\%. This provides evidence that advertising serves to accelerate the consumer's time-to-purchase.

Second, we use a Conditional Logit model to analyze how clicks affect *product choice*, conditional on a purchase occurring. We construct choice sets for each purchasing journey, consisting of the chosen product and a sample of non-chosen alternatives that were impressed upon the user. The utility that user $u$ in journey $j$ derives from product $i$ is modeled as:

$ U_{ij} = \beta_1 \text{WasClicked}_{ij} + \beta_2 \text{Price}_i + \dots + \epsilon_{ij} \quad (4) $

The results are striking. The odds ratio for the $\text{WasClicked}_{ij}$ variable is 408.68 (p < 0.001). It is crucial to note that this massive effect is conditional on a purchase decision already being made; it does not imply an overall lift of this magnitude but rather demonstrates the click's power as a 'kingmaker' in influencing the final choice among a set of considered alternatives. This indicates that, within a set of considered products, a click increases the odds of a product being the one that is ultimately purchased by a factor of over 400. This provides our strongest evidence of the direct persuasive power of sponsored search advertising in influencing consumer choice.

\subsubsection{Measuring Spillover: Brand and Department Halo Effects}

Finally, we quantify the extent to which a click on a specific product generates positive externalities, or "halo effects," for related products. We re-estimate the baseline logistic model, replacing the outcome variable with indicators for whether a product from the same brand or department was purchased. The results, shown in Table \ref{tab:spillover_effects_detailed}, demonstrate significant spillover effects. A click on a product is associated with a 65.5\% increase in the odds of purchasing from the same brand and a 26.0\% increase in the odds of purchasing from the same department, highlighting the broader brand-building and category-development value of advertising.

\begin{table}[h!]
\centering
\caption{Quantifying Spillover (Halo) Effects of Clicks}
\label{tab:spillover_effects_detailed}
\begin{tabular}{lc}
\hline
\textbf{Dependent Variable} & \textbf{Odds Ratio for Clicks (95\% CI)} \\
\hline
Purchase from Same Brand & 1.655*** (1.45 - 1.88) \\
Purchase from Same Department & 1.260*** (1.18 - 1.35) \\
\hline
\multicolumn{2}{l}{\text{Models control for the full set of covariates. *** p<0.001}}
\end{tabular}
\end{table}

\subsubsection{Synthesis of Findings}

In summary, our micro-level analysis provides robust, multi-faceted evidence for the causal impact of sponsored search clicks. We find a strong, positive average treatment effect on both purchase probability and revenue. This effect is, however, non-linear, exhibiting diminishing returns, and is highly heterogeneous across user, journey, and product characteristics. Advanced models demonstrate that clicks not only accelerate the timing of purchases but also exert a powerful influence on product choice itself. Finally, we quantify significant positive spillovers to related products, underscoring the full-funnel impact of advertising. Taken together, these findings resolve the tension between selection and persuasion: while clicks are indeed a strong signal of pre-existing intent (selection), our models demonstrate that they also exert a significant and multi-faceted causal influence on consumer behavior by accelerating purchase decisions, shaping final product choice, and generating valuable brand-level spillovers (persuasion).