% References

\begin{frame}{References: Ghost Ads Methodology}

\small

Johnson, G. A., Lewis, R. A., \& Nubbemeyer, E. I. (2017). Ghost ads: Improving the economics of measuring online ad effectiveness. \textit{Journal of Marketing Research}, 54(6), 867-884.

\vspace{0.5cm}

Lewis, R. A., \& Rao, J. M. (2015). The unfavorable economics of measuring the returns to advertising. \textit{Quarterly Journal of Economics}, 130(4), 1941-1973.

\end{frame}

\begin{frame}{References: Incrementality and Attribution}

\small

Gordon, B. R., Zettelmeyer, F., Bhargava, N., \& Chapsky, D. (2019). A comparison of approaches to advertising measurement: Evidence from big field experiments at Facebook. \textit{Marketing Science}, 38(2), 193-225.

\vspace{0.5cm}

Blake, T., Nosko, C., \& Tadelis, S. (2015). Consumer heterogeneity and paid search effectiveness: A large-scale field experiment. \textit{Econometrica}, 83(1), 155-174.

\end{frame}

\begin{frame}{References: Experimental Design}

\small

Imbens, G. W., \& Angrist, J. D. (1994). Identification and estimation of local average treatment effects. \textit{Econometrica}, 62(2), 467-475.

\vspace{0.5cm}

Angrist, J. D., Imbens, G. W., \& Rubin, D. B. (1996). Identification of causal effects using instrumental variables. \textit{Journal of the American Statistical Association}, 91(434), 444-455.

\end{frame}

\begin{frame}{References: Industry Applications}

\small

Johnson, G. A., Lewis, R. A., \& Reiley, D. H. (2017). When less is more: Data and power in advertising experiments. \textit{Marketing Science}, 36(1), 43-53.

\vspace{0.5cm}

Sahni, N. S. (2015). Effect of temporal spacing between advertising exposures: Evidence from online field experiments. \textit{Quantitative Marketing and Economics}, 13(3), 203-247.

\end{frame}
