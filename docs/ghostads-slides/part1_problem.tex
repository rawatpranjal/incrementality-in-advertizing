% Part 1: The Problem

\begin{frame}[plain]
\begin{center}
\includegraphics[width=0.85\textwidth]{meme.png}
\end{center}
\end{frame}

\begin{frame}{Why Ad Platforms Need to be Incrementality Native}

\vspace{0.3cm}

\large

\begin{itemize}
\item Measuring incrementality = proving value and building trust
  \begin{itemize}
  \item Advertisers distrust last click attribution and run their own tests
  \end{itemize}

\pause

\vspace{0.3cm}

\item Incrementality will be a core offering in retail media
  \begin{itemize}
  \item Molocco, Meta, Google already implementing it
  \item Technology is $\sim$7 years old, pretty mature
  \end{itemize}

\pause

\vspace{0.3cm}

\item Incrementality is not just reporting
  \begin{itemize}
  \item Can be used in budget pacing and ranking algorithms
  \end{itemize}

\end{itemize}

\end{frame}

\begin{frame}{Ad exposure correlates with purchase, but there are confounders}

\pause

\begin{center}
\begin{tikzpicture}[
    node distance=2cm,
    every node/.style={font=\large,align=center},
    arrow/.style={-{Stealth[length=3mm]}, very thick}
]
    \node (intent) [draw, rectangle, minimum width=3cm, minimum height=1cm, fill=BrandAccent!20, line width=2.5pt, rounded corners=4pt]
        {\textbf{User Intent}\\Shopping urgency};
    \node [above=0.3cm of intent, font=\Huge\bfseries] {\textcolor{ErrorRed}{!!}};

    \node (ad) [draw, rectangle, below left=1.8cm and 1.5cm of intent, minimum width=2.8cm, minimum height=1cm, fill=BrandSecondary!20, line width=2.5pt, rounded corners=4pt]
        {\faDesktop\ \textbf{Ad Exposure}};

    \node (purchase) [draw, rectangle, below right=1.8cm and 1.5cm of intent, minimum width=2.8cm, minimum height=1cm, fill=SuccessGreen!20, line width=2.5pt, rounded corners=4pt]
        {\faShoppingCart\ \textbf{Purchase}};

    \pause
    \draw [arrow, BrandAccent] (intent) -- (ad);
    \draw [arrow, BrandAccent] (intent) -- (purchase);

    \pause
    \draw [arrow, dashed, ErrorRed, line width=3pt] (ad) -- node[above, sloped, fill=white] {High Correlation} (purchase);
\end{tikzpicture}
\end{center}

\pause

\visible<5->{
\begin{warningbox}
\centering
\textbf{High correlation overestimates true causal effects}
\end{warningbox}
}

\end{frame}

\begin{frame}{Last-click attribution inflates ROI when intent is high}

\pause

\vspace{0.3cm}

\begin{center}
\Huge
\textcolor{BrandSecondary}{Last-Click Attribution}
\end{center}

\vspace{0.2cm}

\begin{center}
\normalsize
Credit 100\% of conversion to the last ad clicked before purchase
\end{center}

\pause

\vspace{0.3cm}

\begin{columns}
\column{0.48\textwidth}
\begin{center}
\textcolor{SuccessGreen}{\textbf{Why it's popular:}}
\end{center}

\begin{itemize}
\item Easy to implement
\item Simple to explain
\item No experiments needed
\end{itemize}

\pause

\column{0.48\textwidth}
\begin{center}
\textcolor{ErrorRed}{\textbf{The problems:}}
\end{center}

\vspace{0.2cm}

\xmark\ Ignores confounders\\[0.2cm]
\xmark\ What about first/middle clicks?\\[0.2cm]
\xmark\ Advertisers ignore this and run own XPs

\end{columns}

\end{frame}

% STANDOUT FRAME - Keynote style for impact
\begin{frame}[standout]
\vspace{1cm}

\LARGE
\faBuilding\quad \textbf{Industry Leaders Use Ghost Ads}

\vspace{0.8cm}

\normalsize
\begin{center}
Google \quad Yahoo \quad Meta \quad Netflix \quad Criteo

\vspace{0.3cm}

Pandora \quad Roku \quad Nielsen \quad Uber \quad DoorDash

\vspace{0.3cm}

Walmart \quad Catalina \quad CyberAgent \quad Moloco
\end{center}

\end{frame}
