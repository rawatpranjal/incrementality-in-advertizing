% Part 2: What People Usually Try

\begin{frame}{Randomization eliminates confounding}

\pause

\textbf{\large The idea:}

\begin{itemize}
\item Treatment group can see ad
\item Control group blocked from ad
\item Compare average outcomes
\end{itemize}

\pause

\vspace{0.5cm}

\textbf{\large The problem:}

\begin{center}
\begin{tikzpicture}[box/.style={draw, rectangle, rounded corners, minimum width=2.5cm, minimum height=0.8cm, align=center, line width=2pt}]
    \node (assigned) [box, fill=BrandSecondary!20] {10,000\\eligible};
    \node (saw) [box, fill=SuccessGreen!20, right=5cm of assigned] {1,000\\actually saw\\ad};

    \draw [-{Stealth[length=3mm]}, very thick, ErrorRed] (assigned) -- node[above] {\textbf{90\% never saw ad}} (saw);
\end{tikzpicture}
\end{center}

\pause

\visible<4->{
\begin{warningbox}
\centering
\textbf{Dilution problem:} In control, we don't know who would have seen ad.\\
\end{warningbox}
}

\end{frame}

\begin{frame}{Simple user holdouts suffer from dilution}

\pause

\begin{columns}
\column{0.48\textwidth}
\begin{center}
\textcolor{SuccessGreen}{\textbf{Treatment Group}}
\end{center}

\vspace{0.2cm}

\begin{tikzpicture}[scale=0.85]
    \node [draw, rectangle, rounded corners=4pt, fill=BrandSecondary!30, minimum width=2.8cm, minimum height=0.75cm, line width=2.5pt, align=center] at (0,0) {
        {\small \textbf{Saw ad}}\\
        {\tiny 1,000 users}
    };
    \node [draw, rectangle, rounded corners=4pt, fill=black!10, minimum width=2.8cm, minimum height=0.75cm, line width=2.5pt, align=center] at (0,-1.3) {
        {\small \textbf{Never saw ad}}\\
        {\tiny 9,000 users}
    };
\end{tikzpicture}

\vspace{0.15cm}
{\small \textcolor{SuccessGreen}{\checkmark\ We CAN identify}}

\column{0.48\textwidth}
\begin{center}
\textcolor{ErrorRed}{\textbf{Control Group}}
\end{center}

\vspace{0.2cm}

\begin{tikzpicture}[scale=0.85]
    \node [draw, rectangle, rounded corners=4pt, fill=BrandAccent!10, minimum width=2.8cm, minimum height=0.75cm, line width=2.5pt, dashed, opacity=0.6, align=center] at (0,0) {
        {\small \textbf{Would've seen ad}}\\
        {\tiny ??? users}
    };
    \node [draw, rectangle, rounded corners=4pt, fill=black!5, minimum width=2.8cm, minimum height=0.75cm, line width=2.5pt, dashed, opacity=0.6, align=center] at (0,-1.3) {
        {\small \textbf{Would NOT see ad}}\\
        {\tiny ??? users}
    };
\end{tikzpicture}

\vspace{0.15cm}
{\small \textcolor{ErrorRed}{\xmark\ We CANNOT identify}}

\end{columns}

\pause

\vspace{0.2cm}

\visible<3->{
\begin{warningbox}
\centering
Without knowing who would be exposed, can't compare apples to apples.
\end{warningbox}
}

\end{frame}

\begin{frame}{ITT diluted, LATE has high variance}

\pause

\faCalculator\ Intent-to-Treat (ITT) = Avg. outcomes in treatment - avg. outcomes in control\\[0.3cm]

\pause

\faExclamationTriangle\ Estimate is biased towards 0\\[0.3cm]

\pause

\faCalculator\ Local Avg. Treatment Effect (LATE) = $\frac{\text{ITT}}{\text{\% exposed in treatment}}$\\[0.3cm]

\pause

\faExclamationTriangle\ Example: If only 10\% saw ad, multiply variance by 100\\[0.3cm]

\pause

\faExclamationTriangle\ To get sufficient power will have to run XPs for longer\\[0.3cm]

\pause

\vspace{0.3cm}

\begin{warningbox}
\centering
ITT and LATE are both inadequate
\end{warningbox}

\end{frame}

\begin{frame}{Placebo ads fix dilution but contaminate the control group}

\pause

\faLightbulb\ Show control group users a placebo instead of actual ad\\[0.3cm]

\pause

\textbf{\large Result:} Now we CAN identify both groups!

\pause

\vspace{-0.1cm}

\begin{columns}
\column{0.48\textwidth}
\begin{center}
{\small \textbf{Treatment}}\\
\vspace{0.2cm}
\begin{tikzpicture}[scale=0.8]
    \node [draw, rectangle, rounded corners=4pt, fill=BrandSecondary!30, minimum width=2.5cm, minimum height=0.75cm, line width=2pt, align=center] at (0,0) {
        {\small \textbf{Saw ad}}\\
        {\tiny 1,000 users}
    };
    \node [draw, rectangle, rounded corners=4pt, fill=black!10, minimum width=2.5cm, minimum height=0.75cm, line width=2pt, dashed, opacity=0.6, align=center] at (0,-1.3) {
        {\small \textbf{Never saw ad}}\\
        {\tiny 9,000 users}
    };
\end{tikzpicture}
\end{center}

\column{0.48\textwidth}
\begin{center}
{\small \textbf{Control}}\\
\vspace{0.2cm}
\begin{tikzpicture}[scale=0.8]
    \node [draw, rectangle, rounded corners=4pt, fill=BrandAccent!30, minimum width=2.5cm, minimum height=0.75cm, line width=2pt, align=center] at (0,0) {
        {\small \textbf{Saw placebo ad}}\\
        {\tiny 1,000 users}
    };
    \node [draw, rectangle, rounded corners=4pt, fill=black!10, minimum width=2.5cm, minimum height=0.75cm, line width=2pt, dashed, opacity=0.6, align=center] at (0,-1.3) {
        {\small \textbf{Never saw ad}}\\
        {\tiny 9,000 users}
    };
\end{tikzpicture}
\end{center}
\end{columns}

\pause

\vspace{0.1cm}

\textbf{\large Contamination: Placebo may end up changing the behaviour}

\pause

\vspace{0.3cm}

\begin{columns}
\column{0.48\textwidth}
\faHeart\ Charity PSAs reduce spending\\[0.3cm]
\faStore\ Competitor ads affect purchases

\column{0.48\textwidth}
\faFrown\ Irrelevant ads hurt UX\\[0.3cm]
\faExclamationTriangle\ Brand safety risks

\end{columns}

\end{frame}
